\documentclass{ws-rv9x6}
\usepackage{ws-rv-thm}
\usepackage[square]{ws-rv-van}
\usepackage{simplewick}
\usepackage{relsize}
\usepackage{alphalph}

\newif\ifarxivsubmission

\arxivsubmissiontrue

\ifarxivsubmission
  \usepackage{color}
  \definecolor{darkblue}{rgb}{0.1,0.1,.7}
  \usepackage[colorlinks, linkcolor=darkblue, citecolor=darkblue, urlcolor=darkblue, linktocpage]{hyperref}
  \crop[off]{}
  \makeatletter
  \def\@makechapterhead#1{%                                                                                                      
    {\vbox to 110pt{%        %rvs                                                                                              
%   \refstepcounter{chapter}%                                                                                                  
    \def\thefootnote{\@fnsymbol\c@footnote}%                                                                                   
%   \addcontentsline{toc}{chapter}{\outchapter#1}                                                                              
    \vspace*{37pt}      %VSPACE FROM TRIM SIZE                                                                                 
        \parindent\z@\raggedright\reset@font
        {\centering{%{\CNfont Chapter~\thechapter\par}%                                                                         
         \vskip 0.25in
    \vbox{
    \CTfont #1\par
    }\par}\par}\nobreak\vfill}}}%  %rvs
  \makeatother
\else
  \usepackage[]{hyperref}
\fi

\makeindex

%\newcommand\be{\begin{eqnarray}}
%\newcommand\ee{\end{eqnarray}}
\def\be#1\ee{\begin{align}#1\end{align}}
\newcommand\f\phi
\newcommand\cO{\mathcal{O}}
\newcommand\p[1]{\left(#1\right)}
\newcommand\ptl\partial
\newcommand\e\epsilon
\newcommand\<\langle
\renewcommand\>\rangle
\newcommand\Z{\mathbb{Z}}
\newcommand\de\delta
\newcommand\R{\mathbb{R}}
\newcommand\bx{\mathbf{x}}
\newcommand\nn{\nonumber}
\renewcommand\.{\cdot}
\newcommand\x\times
\newcommand\pdr[2]{\frac{\partial #1}{\partial #2}}
\newcommand\s\sigma
\newcommand\SO{\mathrm{SO}}
\newcommand\De{\Delta}
\newcommand\cS{\mathcal{S}}
\newcommand\oo\infty
\newcommand\SU{\mathrm{SU}}
\newcommand\cH{\mathcal{H}}
\newcommand\bn{\mathbf{n}}
\newcommand\bP{\mathbf{P}}
\renewcommand\b\beta
\renewcommand\a\alpha
\newcommand\Tr{\mathrm{Tr}}
\renewcommand\l\lambda
\newcommand\cL{\mathcal{L}}
\newcommand\cD{\mathcal{D}}
\renewcommand\th{\theta}
\newcommand\tl[1]{\widetilde{#1}}
\newcommand\GeV{\mathrm{GeV}}
\newcommand\g\gamma
\newcommand\arXiv[1]{{\tt arXiv:#1}}
\newcommand\z\zeta
\renewcommand\hat[1]{\widehat{#1}}
\newcommand\bs{\mathbf{s}}
\newcommand\draftnote[1]{{\color{blue} #1}}
\newcommand\C{\mathbb{C}}

\newcommand\stoplecture{
\stepcounter{LectureCounter}
\hfill \draftnote{\it End of lecture \theLectureCounter}
}

\newcounter{LectureCounter}



\newtheorem{exercise}{Exercise}[section]
\let\Exercisefont\itshape
\def\Exerciseheadfont{\bfseries}

\begin{document}

\ifarxivsubmission
\chapter{Phys 229ab Advanced Mathematical Methods:\ \ \\
Conformal Field Theory
}\label{ch:bootstrap}
\else
\chapter{The Conformal Bootstrap}
\fi

\author[David Simmons-Duffin]{David Simmons-Duffin}

\address{Caltech,
dsd@caltech.edu}

\begin{abstract}
Notes for Physics 229, 2017-2018. These notes are in progress. They were last updated on \today. In a few places, there are reminders \draftnote{in blue} to the author to add additional discussion/comments. There may also be empty citations and missing figures. Some of the discussion is not original and is borrowed from various sources. I have tried to indicate with a footnote when a section closely follows a particular reference.
\end{abstract}
\body

\makeatletter
\newalphalph{\alphmult}[mult]{\@alph}{26}
\makeatother
\renewcommand{\thefootnote}{\arabic{footnote}}

\ifarxivsubmission
  \thispagestyle{empty}
  \makeatletter
  \renewcommand*\l@chapter[2]{}
  \renewcommand*\l@schapter[2]{}
  \renewcommand*\l@author[2]{}
  \makeatother
  \newpage
  \setcounter{page}{1}
  \setcounter{tocdepth}{3}
  \smalltoc
  \tableofcontents
\fi

\ifarxivsubmission
  \newpage
\fi


\section*{Resources}

These introductory notes are heavily based on Silviu Pufu's Bootstrap 2017 lectures \cite{Pufu2017}, John Cardy's book {\it Scaling and Renormalization in Statistical Physics\/} \cite{Cardy:1996xt}, Slava Rychkov's EPFL lectures on CFT \cite{Rychkov:2016iqz}, and John McGreevy's lectures on QFT \cite{McGreevy2017}.


\section{Introduction}

\subsection{QFT and emergent symmetry}

Quantum Field Theory is a universal language for theoretical physics. It shows up in many different settings, for example
\begin{itemize}
\item statistical physics,
\item condensed matter physics,
\item particle physics (SM and beyond),
\item string theory/holography.
\end{itemize}

The microscopic physics in all of these settings can be quite complicated. However, often the macroscopic physics displays extra ``emergent" symmetries that can help us do computations.

For example, in condensed matter physics, we are interested in describing a material made up of atoms with some lattice spacing $a$.\footnote{We ignore the possibility of lattice defects for the moment.} We refer to the detailed lattice system as the ``microscopic theory." Quantum field theory is a good description at distances much larger than the lattice spacing $x\gg a$, or equivalently energy/momenta much lower than the UV cutoff $\Lambda_{UV}=1/a$.\footnote{QFTs that are valid below a UV cutoff are often called ``effective field theories" (EFTs).} At these large scales, the discrete translation symmetry of the lattice becomes a continuous symmetry.

Similarly, in statistical physics, particle physics, and string theory, QFT can describe distance scales much larger than the characteristic scales of the microscopic theory.

Statistical systems are described by QFTs in Euclidean signature, e.g.\ on $\R^d$. Such QFTs capture properties of the equilibrium state. By contrast, condensed matter and particle systems are described by QFTs in Lorentzian signature, e.g.\ on $\R^{d-1,1}$. Such QFTs encode time-dependent quantum dynamics.

We will be interested in QFTs with rotational symmetry, by which we mean $\SO(d)$ symmetry in Euclidean signature and $\SO(d-1,1)$ symmetry in Lorentzian signature. In particle physics and string theory, this symmetry is built into the microscopic theory. However in lattice systems, rotational symmetry must be emergent. This means that correlation functions become rotationally invariant in the limit of large distances, even though microscopic correlation functions are not rotationally-invariant.\footnote{Emergent rotational symmetry is very familiar: we� often cannot determine the orientation of a microscopic lattice using macroscopic observations. Some examples of materials {\it without\/} emergent $\SO(d)$ symmetry are crystals like salt. A very exotic example is the Haah code \cite{}.} In particular, for condensed matter systems, the effective ``speed of light" associated with $\SO(d-1,1)$-invariance is an emergent property and has nothing to do with the speed of actual light. (We will see some explicit examples later.)

Under general conditions, $\SO(d)$-invariant Euclidean QFTs are in one-to-one correspondence with $\SO(d-1,1)$-invariant Lorentzian QFTs. The map between them is called Wick rotation, and we will discuss it in detail. Because of this correspondence, we can focus mostly on Euclidean QFTs, and later understand Lorentzian QFTs by Wick rotating what we learned in Euclidean signature.

\subsection{The mass gap and critical points}

So far, we are interested in theories with Poincare symmetry
\be
G_\mathrm{Poincare} &= \R^d \rtimes \SO(d).
\ee
From the point of view of long-distance physics, the most important property of a Poincare-invariant theory is its mass gap 
\be
m_\mathrm{gap} &= E_1-E_0,
\ee
where $E_0,E_1$ are the energies of the ground state and first excited state, respectively.\footnote{Some QFTs with topological order can have multiple degenerate ground states.} Theories with $m_\mathrm{gap}>0$ are called ``gapped." 

To understand why $m_\mathrm{gap}$ is important, let us study a two-point function of a scalar local operator $\f(x)=\f(x^0,\mathbf{x})$. We  demand that $\f(0)$ have vanishing vacuum expectation value by subtracting off an appropriate multiple of the unit operator. By Poincare invariance, it suffices to consider $\<0|\f(x^0,\mathbf{0}) \f(0)|0\>$ with $x^0>0$. The two-point function is then given by
\be
\label{eq:completeset}
\<0|\f(x^0,\mathbf{0})\f(0)|0\> &= \<0|\f(0) e^{-H x^0} \f(0)|0\> \nn\\
&= \sum_\psi |\<0|\f(0)|\psi\>|^2 e^{-E_\psi x^0}.
\ee
Here, $H$ is the Hamiltonian with the vacuum energy subtracted off, and $\psi$ runs over an orthonormal basis of eigenstates of $H$. To get the right-hand side, we have used $\f(x^0,\mathbf{0}) = e^{H x^0} \f(0,\mathbf{0}) e^{-H x^0}$ and  $H|0\>=0$.\footnote{Note that the Euclidean time-evolution operator is $e^{-H x^0}$ as opposed to the familiar $e^{-iHt}$ in Lorentzian signature. They are related by Wick rotation $x^0 = i t$. We will discuss this in much more detail in later sections.}

The key point is that the operator $e^{-H x^0}$ exponentially damps states with energy $E_\psi\gg 1/x^0$.
 %%We have the Kallen-Lehmann spectral representation
%%\be
%%\<\f(x)\f(y)\> &= \int dm^2 \rho(m^2) G_F(x-y,m),
%%\ee
%%where $G_F(x,m^2)$ is the Feynman propagator with mass $m$. Here, $\rho(m^2)\geq 0$ is an effective density of states with mass $m$. If we demand that $\f(0)$ have vanishing vacuum expectation value (by subtracting off an appropriate multiple of the unit operator), then the support of $\rho(m^2)$ is contained in $[m_\mathrm{gap}^2,\oo)$.
%%
%% For simplicity, consider $d=3$ where the Feynman propagator simplifies
%%\be
%%G_F(x,m) &= \frac{1}{4\pi |x|} e^{-m|x|} \qquad (d=3).
%%\ee
%The key point is that $G_F(x,m)$ falls off exponentially as $e^{-m|x|}$ in Euclidean signature. At large $|x|$, the two-point function is then dominated by contributions with the smallest $m$, and thus falls off as $e^{-m_\mathrm{gap} |x|}/|x|$ if $m_\mathrm{gap}$ is nonzero.\footnote{It could fall faster if the operator $\f(x)$ has zero amplitude to create the lightest state in the theory.}
%
%A more general way to see the exponential falloff is to write the two-point function as
%\be
%\<0|\f(x^0,\mathbf{0}) \f(0)|0\> &= \<0|\f(0)e^{-H x^0} \f(0)|0\>,
%\ee
%where $H$ is the Hamiltonian (with the ground state energy subtracted off), and we have taken the points to be separated only in the $x^0$ direction for simplicity.\footnote{Note that in Euclidean theories, the time-evolution operator is $e^{-H x^0}$ as opposed to $e^{-iHt}$ familiar from quantum mechanics. We will discuss this in much more detail later in the course.} 
At large $x^0$, the correlator is dominated by $\psi$ with the smallest nonzero eigenvalue of $H$, which is $m_\mathrm{gap}$.\footnote{Note that the vacuum does not contribute as an intermediate state because we have demanded $\<0|\f(0)|0\>=0$.)}

Thus, when $m_\mathrm{gap}$ is nonzero, correlation functions of local operators fall off at least as fast as $e^{-|x|/\xi}$, where $\xi\equiv 1/m_\mathrm{gap}$ is called the ``correlation length." Generic statistical and condensed matter systems have microscopic correlation lengths $\xi \sim a$, or equivalently $m_\mathrm{gap} \sim \Lambda_{UV}$. At long distances, they are described by QFTs whose local correlation functions vanish, called topological quantum field theories (TQFTs).

%Generic statistical and condensed-matter systems are gapped. In such systems, correlations at scales $x\gg \l$ are effectively zero.\footnote{Gapped theories are described by topological quantum field theories (TQFTs) and are interesting in their own right. However, we will not focus on TQFTs in this course.}

 However, sometimes by tuning parameters in the microscopic Hamiltonian, we can make $m_\mathrm{gap}$ much smaller than $\Lambda_{UV}$, and even arrange for $m_\mathrm{gap}$ to vanish. Points in parameter space where $m_\mathrm{gap}=0$ are called critical points.
At a critical point, the system experiences a phase transition, and develops nonzero correlations at arbitrarily long distances.\footnote{If the Standard Model were like a generic condensed matter system, we might expect $m_\mathrm{gap}$ to be close to the UV cutoff, which is perhaps the GUT scale $10^{15}\GeV$ or Planck scale $10^{18}\GeV$. The hierarchy problem is the problem of explaining why the Standard Model is so close to a critical point.}

Long-distance correlation functions at critical points have no intrinsic length-scale because all memory of dimensionful microscopic quantities (like the lattice spacing $a$) disappears when distances become arbitrarily large. For example, critical two-point functions behave as pure power laws
\be
\<\f(x)\f(0)\> &= \frac{C}{|x|^{2\De}} \qquad(\textrm{critical point, }x\gg a),
\ee
where $C$ and $\De$ are constants depending on $\phi$. The quantity $\De$ is called the scaling dimension of $\f$.

\subsection{Scaling and conformal symmetry}

A more precise way to state the lack of an intrinsic length scale is to say that theories with $m_\mathrm{gap}=0$ have an emergent symmetry under rescaling
\be
\label{eq:rescalingsymmetry}
x^\mu \to \l x^\mu\qquad (\l>0).
\ee
Under very general conditions (that we will discuss), critical points also display less obvious emergent symmetries called conformal transformations. A conformal transformation $x\to x'(x)$ is a map that looks like a rotation and rescaling near each point,
\be
\pdr{x'^\mu}{x^\nu} &= \Omega(x) R^{\mu}{}_\nu(x),\qquad R^{\mu}{}_\nu\in \SO(d).
\ee
An example is a special conformal transformation,
%a less-obvious emergent symmetry under special conformal transformations
\be
\label{eq:specialconformalsymmetry}
x^\mu \to \frac{x^\mu - b^\mu x^2}{1-2 b\.x + b^2 x^2}\qquad(b\in \R^d).
\ee
In 2-dimensions, there are more exotic examples, like the one pictured in figure~\ref{}.\footnote{We will give a precise definition for what it means for a theory to be invariant under transformations like (\ref{eq:rescalingsymmetry}) and (\ref{eq:specialconformalsymmetry}) later in the course.}

Here is some rough intuition for why critical points display conformal symmetry. We know that a critical theory is invariant under rescalings and rotations. If the theory is also local, in the sense that degrees of freedom at a point only interact directly with other degrees of freedom at nearby points, then the theory should also be invariant under transformations that locally look like a rescaling and rotation. This is the defining property of a conformal transformation. Turning this rough intuition into a theorem is a difficult problem (that we will discuss in more detail later). However, it seems to be true in a wide class of systems.

QFTs that are invariant under conformal symmetry are called conformal field theories (CFTs). To summarize, CFTs describe critical points where $m_\mathrm{gap}=0$. One can also understand the neighborhood of a critical point where $m_\mathrm{gap}$ is nonzero (but still $m_\mathrm{gap} \ll \Lambda_{UV}$) by studying perturbations of the associated CFT.

\subsection[Examples of critical points]{Examples of critical points\footnote{Sources for this section: Pufu \cite{Pufu2017}, Cardy \cite{Cardy:1996xt}.}}


So far our discussion has been very abstract, so let us introduce some examples. 
%Through examples, we will try to explain
%\begin{itemize}
%\item Why statistical systems are described by Euclidean QFT,
%\item How different microscopic theories are described by the same CFT (universality),
%\item How to tune $m_\mathrm{gap}\to 0$ in a solvable model,
%\item How to study (some) CFTs using conventional QFT tools like perturbation theory.
%\end{itemize}
One of our goals will be to infer from examples a set of axioms that CFTs should satisfy. We will study these axioms in the next part of the course. Another goal will be to be more precise about how and why statistical and condensed matter systems are described by quantum field theory, and how critical points come about.

\subsubsection{Magnets}

Our first examples of critical points occur in magnets.  Given a magnet with temperature $T$, we can apply a magnetic field $\vec{H}$ and measure the magnetization $\vec{M}$. There are three main types of magnets in 3-dimensions, which are distinguished by their symmetries:\footnote{Here, we mean non-spacetime symmetries, usually called ``global" or ``flavor" symmetries. The emergent spacetime symmetry group is still the Poincare group, or the conformal group at the critical point.}
\begin{itemize}
\item {\it Uni-axial magnet}: individual magnetic moments $\vec\mu$ are confined to lie along a fixed axis. Uni-axial magnets have an $O(1)=\Z_2$ symmetry under which $\vec H \to -\vec H$ and $\vec M \to -\vec M$.

\item {\it XY magnet}: magnetic moments $\vec\mu$ are oriented in a plane. Such magnets have an $O(2)$ symmetry under which $\vec H$ and $\vec M$ rotate in the plane.

\item {\it Heisenberg magnet}: magnetic moments $\vec \mu$ are unconstrained. Such magnets have $O(3)$ symmetry under which $\vec H$ and $\vec M$ transform in the vector representation.
\end{itemize}
For the moment, we will focus on the simplest case of uni-axial magnets. We denote the projections of $\vec H,\vec M$ onto the appropriate axis by $H,M$.

In experiments, we observe the following. There exists a ``critical temperature" $T_c$, such that
\begin{itemize}
\item For $T<T_c$, the preferred state of the magnet has nonzero magnetization $M\neq 0$ when $H=0$. In other words, the $\Z_2$ symmetry is spontaneously broken.
\item For $T>T_c$, the magnet has $M=0$ when $H=0$, i.e.\ the $\Z_2$ symmetry is unbroken.
\end{itemize}
The corresponding phase diagram is pictured in figure~\ref{}. The point 
\be
H = 0,\ T=T_c
\ee
is a critical point, and is described by a CFT at long distances. To reach the critical point, we must tune two parameters: $H$ and $T$. Tuning $H=0$ is easy because that is where the microscopic theory has $\Z_2$ symmetry. However, the value of $T_c$ depends on the specific material.

In more detail, the behavior of the magnetization in different phases is shown in figure~\ref{}. Close to $T_c$, observables exhibit so-called ``scaling" behavior, characterized by various critical exponents. Let us define the dimensionless couplings 
\be
\label{eq:dimensionlesscouplings}
t \equiv \frac{T-T_c}{T_c},\qquad h=\frac{H}{k_B T}.
\ee
Some examples of critical exponents are
\begin{itemize}
\item $\a$: the heat capacity at $h=0$ behaves as
\be
C &= \frac{\ptl^2 F}{\ptl T^2} \propto |t|^{-\a}.
\ee
(Here $F$ is the free-energy.)

\item $\b$: the spontaneous magnetization behaves as
\be
\lim_{H\to 0^+} M &\propto (-t)^\b.
\ee

\item $\gamma$: the zero-field susceptibility behaves as
\be
\chi &= \left.\pdr{M}{H}\right|_{H=0} \propto |t|^{-\g}.
\ee

\item $\de$: the magnetization at $T=T_c$ behaves as
\be
|M|&\propto|h|^{1/\de}.
\ee

\item $\nu$ and $\eta$: the correlation length $\xi$ can be measured by studying a two-point correlation functions of spins 
\be
G(x) &= \<s(x)s(0)\> - \<s(0)\>^2.
\ee
Away from the critical point, $G(x)\sim e^{-|x|/\xi}$ decays exponentially. However, as $t\to 0$, the correlation length diverges as
\be
\label{eq:correlationlengthexponent}
\xi &\propto |t|^{-\nu}.
\ee
Equivalently, the mass-gap goes to zero as $m_\mathrm{gap}\propto |t|^\nu$.
Precisely at $t=0$, the two-point function takes the form
\be
G(x) &\propto \frac{1}{|x|^{d-2+\eta}},
\ee
i.e.\ the spin operator has dimension $\Delta_s = \frac{d-2}{2}+\frac \eta 2$.
\end{itemize}

Don't worry, I can't keep track of all these critical exponents either. We will see shortly that all of this behavior can be explained using effective field theory, scaling symmetry, and dimensional analysis.

Now, here is an amazing fact:
\begin{claim}
We find the same critical exponents in many different uni-axial magnets, regardless of what material they're made of.
\end{claim}
In fact, critical uni-axial magnets are all described by the same scale-invariant QFT at long distances. This phenomenon is called ``critical universality."

\subsubsection{Liquid-vapor transitions}

Other critical points appear in liquid-vapor transitions. For example, the phase-diagram of water is pictured in figure~\ref{}. Near room temperatures and pressures, there is a sharp distinction between the liquid and gas phases. However, at  higher temperatures and pressures, the distinction between liquid and gas disappears at a critical point $(T_c,P_c)$.  For example, in water $T_c=647\,\mathrm{K}$, $P_c=374\,\mathrm{Atm}$.

Note that the critical points of magnets and water are both obtained by tuning two parameters. Comparing neighborhoods of the critical points in figures~\ref{} and \ref{}, we can make the following rough analogy between water and magnets:
\be
P-P_c &\sim H, \nn\\
\rho-\rho_c &\sim M.
\ee
where $\rho$ is the density and $\rho_c$ is the critical density.\footnote{Water does not have a microscopic $\Z_2$ symmetry, but it turns out that one emerges near the critical point. Roughly speaking, the $\Z_2$ switches the liquid (high-density) and gas (low-density) phases. To make a more precise analogy, we should identify $M$ with the combination of $\rho$ and $P$ that flips sign under the emergent $\Z_2$.}

In measurements of critical water, we again observe scaling behavior. For example, the heat capacity behaves as
\be
C &\sim |t|^{-\a},
\ee
where $t$ is again given by (\ref{eq:dimensionlesscouplings}).
Additionally, the difference in density between the liquid and gas phases behaves as
\be
\rho_\mathrm{liquid}-\rho_\mathrm{gas} &\sim (-t)^\b.
\ee
Amazingly,
\begin{claim}
Water and other liquid-vapor transitions have precisely the same critical exponents as uni-axial magnets.
\end{claim}
We say that liquid-vapor transitions are in the same ``universality class" as uni-axial magnets.

\stoplecture

\subsubsection{The Ising model}

The Ising model is a simplified model of a magnet that is still rich enough to capture its critical behavior. Its degrees of freedom are classical spins taking values $\pm 1$, with one spin for each site on a cubic lattice\footnote{The Ising model can also be formulated on other lattices, and in many cases these models lie in the same universality class.}
\be
s_i \in \{\pm 1\},\qquad i \in \Z^d.
\ee
The partition function is a sum over all configurations of spins, with Boltzmann weights that mimic the interactions between physical spins in a magnet,
\be
Z(K,h) &= \sum_{\{s_i\}} e^{-S[s]}, \nn\\
S[s] &= -K \sum_{\<ij\>} s_i s_j - h \sum_j s_j.
\ee
The notation $\<ij\>$ means that $i,j$ are neighboring lattice sites. (You can think of $K$ as the product $\b J$ where $\b$ is the inverse temperature and $J$ is the spin-spin interaction.) We can also compute correlation functions by inserting spins into the sum
\be
\<s_{i_1} \cdots s_{i_n}\> &= \frac{1}{Z}\sum_{\{s_i\}} s_{i_1} \cdots s_{i_n} e^{-S[s]}.
\ee
The ``magnetization" is proportional to the one-point function $\<s_i\>$ (which is independent of $i$).

When $K$ is positive, the term in $S[s]$ proportional to $K$ makes spins want to align. This term has a $\Z_2$ symmetry under which $s_i\to -s_i$ for all $i$. The term proportional to $h$ breaks this $\Z_2$ symmetry and causes spins to preferentially have the same sign as $h$. Both of these effects compete against statistical fluctuations.

In dimension $d \geq 2$, the Ising model famously exhibits a critical point at a special value $K=K_c$ and $h=0$. For $K>K_c$, spins spontaneously align and break the $\Z_2$ symmetry. For $K<K_c$, statistical fluctuations cause the spins to randomize and the $\Z_2$ is unbroken.

The Ising critical point displays precisely the same long-distance correlation functions and critical exponents as real uni-axial magnets and liquid-vapor systems. In fact, for our purposes, we can think of the Ising model as yet another physical system in the same universality class as these other systems. Its main distinguishing feature is that it is much simpler than e.g.\ actual water, and thus much easier to study.\footnote{There are many other abstract statistical lattice models in the same universality class as the Ising model, for example compass models \arXiv{1303.5922} and the Blume-Capel model \arXiv{1711.10946}. The Ising model on a cubic lattice is arguably the simplest, however the Blume-Capel model has some features that make it even better for simulation.} By studying the mathematics of the Ising model, we will be able to understand gaplessness and a few QFT/CFT axioms in a simple way. By the power of critical universality, quantities we compute in the abstract Ising model at $K=K_c$ agree exactly with interesting quantities in real physical systems.

\subsubsection{Continuum $\f^4$ theory}

The Ising lattice model is a good starting point for understanding why classical statistical systems are described by QFT at long distances. We can think of each configuration of spins as a map
\be
s: \Z^d \to \{\pm 1\},
\ee
and the partition function is a sum over all such maps. This is a discrete version of a path integral.

Typically, in QFT we integrate over continuous functions on a continuous space, e.g.\ 
\be
\f : \R^d \to \R.
\ee
If we study correlation functions of the Ising model at long distances (much larger than the lattice spacing), it's not hard to imagine that we can approximate the lattice as continuous $i\in\Z^d \to x\in\R^d$. Similarly, we can replace individual spins $s_i$ with average densities of spins $\f(x)$ in small neighborhoods, so that the effective spin at a point becomes a real number. 

This suggests that the partition sum of the Ising model might be related to the path integral for a scalar field on $\R^d$, at least at long distances. That is, perhaps we can take
\be
\Z^d &\to \R^d\nn\\
s_i &\to \phi(x) \nn\\
\sum_{\{s_i\}} &\to \int D\phi \nn\\
e^{-S[s]} &\to e^{-S[\phi]},
\ee
for some continuum action $S[\f]$, without changing the long-distance behavior of correlation functions too much.

There is a non-rigorous procedure called ``taking the continuum limit" of a lattice model that tries to justify these replacements. However, it involves several uncontrolled approximations, and it is perhaps more honest to just study a continuum scalar theory and compare its behavior with the Ising model.

For concreteness, let us focus on 3-dimensions. The simplest interacting theory of a scalar field in 3d has Euclidean action
\be
\label{eq:continuumphifour}
S[\f] &= \int d^3 x \p{\frac 1 2 (\ptl \phi)^2 + \frac 1 2 m^2 \f^2 + g \f^4}.
\ee
Note that $\f$ has mass-dimension $1/2$, so that $m$ and $g$ both have mass-dimension $1$. Thus, $\f^2$ and $\f^4$ are relevant interactions, and we can think of this theory as an RG flow from the free-boson at high energies to something else at low energies.

Let us map out the phase diagram of this theory. Because $m$ and $g$ are the only dimensionful parameters, only their ratio matters.\footnote{To be more precise, there is also a UV cutoff $\Lambda$ that depends on how the theory is regularized. The quartic term gives rise to a linear divergence proportional to the $\f^2$ term, so  physical masses are given by $m_\mathrm{phys}^2 = m^2 + \a g \Lambda$, with $\a$ a dimensionless number. (In some regularization schemes, like dimensional regularization, the linear divergence vanishes.) However, shifting $m^2$ is the only way the cutoff can enter a physical observable. Thus, our discussion becomes correct after the replacement $m^2 \to m^2_\mathrm{phys}$.}   Suppose first that $m^2/g^2 \gg 0$ is large and positive. In this limit, the theory has a single massive vacuum with mass approximately $m_\mathrm{gap} \approx m$. Now suppose $m^2/g^2 \ll 0$ is very negative. In this case, the theory has two very massive vacuua that spontaneously break the $\Z_2$ symmetry, with $m_\mathrm{gap}\approx \sqrt {-2m^2}$.

If we start at very large $|m^2|$ (in either phase) and decrease $|m^2|$, $m_\mathrm{gap}$ decreases too. There exists a critical ratio $m^2/g^2 = r_c$ where $m_\mathrm{gap}\to 0$ and the theory is described by a nontrivial CFT. One can justify this claim using the $\e$-expansion and other perturbative techniques, as we will discuss, or from numerical simulations.

The phase diagram we have just described is the same as the Ising phase diagram. Indeed $\f^4$-theory is also in the Ising universality class.

Near the critical point, the $\f^4$ interaction cannot be treated as a small perturbation. More explicitly, because the coupling constant $g$ has mass-dimension 1, perturbation theory is really an expansion in $g x$, where $x$ is the characteristic length scale of the observable we are computing. In the long-distance limit $x\to \oo$ (where the CFT emerges), this expansion breaks down. Thus, the Ising CFT cannot be studied with traditional perturbation theory. We will later discuss two different perturbative (but uncontrolled) expansions for the Ising CFT: the $\e$-expansion and the large-$N$ expansion.

\begin{figure}
\begin{center}
\includegraphics[width=0.55\textwidth]{rgflows-ising.jpg}
\end{center}
\caption{Many  microscopic theories can flow to the same IR CFT\@. We say that the theories are IR equivalent, or IR dual. \label{fig:rgflows}}
\end{figure}

The theory with action (\ref{eq:continuumphifour}) and $m^2/g^2=r_c$ is equivalent to the free boson at short distances (high energies/UV) and the Ising CFT at long distances (low energies/IR). The free boson is itself a CFT, so we have an RG flow from one CFT to another (figure \ref{fig:rgflows}). This construction, where we start with a UV CFT and  perturb it so that it flows to an IR CFT, is one possible definition of a general QFT. In this definition, we must allow for the possibility that the IR theory is gapped, in which case $\mathrm{CFT}_\mathrm{IR}$ is a TQFT which is technically a special case of a CFT (where all local correlation functions are zero).

An RG flow from the free boson is perhaps the cleanest theoretical construction of the Ising CFT. It shows that the Ising CFT inherits properties of a continuum quantum field theory. For example, $\f^4$-theory has rotational invariance and $\Z_2$ symmetry, so the Ising CFT does too. The Ising lattice model is easier to simulate, but it does not have microscopic rotational symmetry. Meanwhile, water has rotational symmetry but no microscopic $\Z_2$ symmetry. We will learn a lot about the Ising CFT by going back and forth between different microscopic realizations.

\subsubsection{Other universality classes}

We have seen several different systems that fall into the Ising universality class. However, not every critical point is described by the Ising CFT. Another important class of 3d CFTs are the $O(N)$ models. These can be described as the critical point of a theory of $N$ bosons $\f_i$ ($i=1,\dots,N$) with a quartic interaction that respects an $O(N)$ global symmetry
\be
S &= \int d^3 x \p{\frac 1 2 \sum_i \ptl_\mu \f_i \ptl^\mu \f_i + \frac 1 2 m^2 \sum_i \f_i \f_i + g \p{\sum_i \f_i \f_i}^2}.
\ee
XY magnets are in the same universality class of the $O(2)$ model and Heisenberg magnets are in the same universality class of the $O(3)$ model.

\draftnote{Discuss non-uniqueness of UV Lagrangian description. An example is particle-vortex duality between the gauged $O(2)$ model and the $O(2)$ model. Distinguish between IR duality and exact duality.}

But this is just the tip of the iceberg --- there is a huge zoo of different types of critical points that we can construct in a variety of ways. One of our goals will be to better understand this zoo.

\section[Path integrals, quantization, and the quantum-classical correspondence]{Path integrals, quantization, and the quantum-classical correspondence\footnote{Sources for this section: McGreevy \cite{McGreevy2017}.}}

In this section, we explain in more detail why classical statistical systems are described by Euclidean quantum field theory, and also how they are related to quantum condensed-matter systems via Wick rotation. Along the way, we will introduce some important technical concepts in QFT, like cutting-and-gluing rules for path integrals and the idea of quantization.

\subsection{The 1d Ising model and the transfer matrix}

Let us start with the Ising lattice model in 1-dimension. For concreteness, we will study the theory on a periodic lattice with length $M$, so the spins $s_i$ are labeled by $i\in \Z_M$.  The partition function is given by
\be
Z_1 &= \sum_{\{s_i=\pm 1\}} e^{-S[s]} \nn\\
S[s] &= -K \sum_{i=1}^{M} s_i s_{i+1} - h \sum_{i=1}^M s_i.
\ee
We refer to $S[s]$ as the ``action," even though it is equal to $\b H$, where $H$ is the classical Hamiltonian. This is because we would like to reserve the word Hamiltonian for a completely different object that will appear shortly.

As mentioned in the introduction, the partition sum should be thought of as a discrete version of a 1-dimensional path-integral. This 1-dimensional path integral can be computed by relating it to a 0-dimensional quantum theory. This is an example of the notion of {\it quantization}.

The key idea is to build up the partition sum by moving along the lattice site-by-site. Forget about periodicity for the moment, and consider the contribution to the partition function from spins $j<i$ for some fixed $i$,
\be
Z_\mathrm{partial}(i,s_i) &= \sum_{\{s_j:j<i\}} e^{K\sum_{j < i} s_j s_{j+1} + h \sum_{j< i} s_j}.
\ee
Because of the interaction term $s_{i-1}s_{i}$, we cannot do the sum over $\{s_1,\dots, s_{i-1}\}$ without specifying the spin $s_i$. Thus, we have a function of $s_i$. In short, $Z_\mathrm{partial}(i,s)$ is the partition function of the theory on the lattice $1\dots i$, with fixed boundary condition $s$ at site $i$.\footnote{We must also impose some boundary condition at site $1$. The precise choice is not important for this discussion, so we have left it implicit.}

Note that $Z_\mathrm{partial}(i+1,s_{i+1})$ can be related to $Z_\mathrm{partial}(i,s_{i})$ by inserting the remaining Boltzmann weights that depend on $s_i$ and performing the sum over $s_i=\pm 1$,
\be
\label{eq:onestepsum}
Z_\mathrm{partial}(i+1,s_{i+1}) &= \sum_{s_i = \pm 1} T(s_{i+1},s_i) Z_\mathrm{partial}(i,s_{i}),
\ee
where
\be
T(s_{i+1},s_i) &\equiv e^{K s_i s_{i+1} + h s_i}.
\ee

The key step is to recognize (\ref{eq:onestepsum}) as a discrete version of the Schrodinger equation in a 2-dimensional Hilbert space $\mathcal{H}$. This Hilbert space has basis $|s\>=|\pm 1\>$. The $T(s',s)$'s are elements of a $2\x 2$  matrix $\hat{T}$ acting  on $\mathcal{H}$
%We can think of $T(s',s)$ as elements of a $2\x 2$ matrix and $Z_\mathrm{partial}(s)$ as components of a vector in a 2-dimensional Hilbert space,
\be
T(s',s) &= \<s'|\hat{T}|s\>,\qquad \hat{T} =
\begin{pmatrix}
e^{K+h} & e^{-K-h} \\
e^{-K+h} & e^{K-h}
\end{pmatrix},
\ee
and $Z_\mathrm{partial}(i,s)$ are the components of a vector $|\Psi_i\>\in \mathcal{H}$,
\be
Z_\mathrm{partial}(i,s) &= \<s|\Psi_i\>.
\ee
In this notation, (\ref{eq:onestepsum}) becomes
\be
\label{eq:discreteshrodinger}
|\Psi_{i+1}\> &= \hat{T} |\Psi_i\>.
\ee
The matrix $\hat{T}$ is called the ``transfer matrix", and it plays the role of a discrete time-translation operator. Here, $i$ should be thought of as a discrete Euclidean time coordinate.

To be explicit, the (integrated) Schrodinger equation in a quantum theory in Euclidean time is
\be
|\Psi(\tau+\Delta\tau)\> &= e^{-\De\tau \hat H} |\Psi(\tau)\>,
\ee
where $\tau$ is the Euclidean time coordinate, $\De\tau$ is some time-step, and $\hat H$ is the quantum Hamiltonian. Thus,  the $1$-dimensional Ising lattice model is equivalent to a 2-state quantum theory with Hamiltonian
\be
\hat H &= -\frac{1}{\Delta\tau} \log \hat T.
\ee

When the lattice is periodic, the partition function is related to the transfer matrix by
\be
\label{eq:partitionfunctionastransfermatrix}
Z_1 %&= \sum_{\{s_i\}} T(s_M, s_{M-1}) T(s_{M-1},s_{M-2}) \cdots T(s_1,s_M) \nn\\
&= \sum_{\{s_i\}} \<s_M|\hat{T}|s_{M-1}\>\<s_{M-1}|\hat{T}|s_{M-2}\> \cdots \< s_1|\hat{T}|s_M\> \nn\\
&= \Tr(\hat{T}^M).
\ee
This is now easy to evaluate by diagonalizing $\hat T$,
\be
\Tr(\hat{T}^M) &= \l_+^M + \l_-^M,
\ee
where
\be
\l_\pm &= e^K \cosh h \pm \sqrt{e^{2K} \sinh^2 h + e^{-2K}} \nn\\
&\to \begin{cases}
2\cosh K \\
2\sinh K
\end{cases}\qquad (\textrm{when }h=0).
\ee

In the thermodynamic limit $M\to\oo$, the partition function is dominated by the larger eigenvalue
\be
Z_1 &=\l_+^M \p{1+\p{\frac{\l_-}{\l_+}}^M} \approx \l_+^M.
\ee
In quantum mechanical language, the state with the largest eigenvalue of $\hat T$ has the smallest eigenvalue of $\hat H$ --- i.e.\ it is the ground state, and we should call it $|0\>$. We have shown that the ground state dominates the thermodynamic limit. Contributions from the excited state are exponentially suppressed in the energy gap times the size of the system
\be
\p{\frac{\l_-}{\l_+}}^M &= e^{-(M\Delta\tau) m_\mathrm{gap}}, \nn\\
m_\mathrm{gap} &\equiv - \frac 1 {\Delta \tau} \log (\l_-/\l_+).
\ee

\stoplecture

We can also use the transfer matrix to compute correlation functions. For example, consider the two-point function $\<s_{i_1} s_{i_2}\>_{\Z_M}$ where the subscript $M$ indicates that we are on a periodic lattice with length $M$. Suppose $i_1>i_2$. We have
\be
\<s_{i_1} s_{i_2} \>_{\Z_M} &= \frac 1 {Z_1} \sum_{\{s_i\}} \<s_M|\hat{T}|s_{M-1}\> \cdots \<s_{i_1+1}|\hat{T}|s_{i_1}\> s_{i_1} \<s_{i_1}|\hat{T}|s_{i_1-1}\>\cdots \nn\\
& \qquad \x\cdots\<s_{i_2+1}|\hat{T}|s_{i_2}\> s_{i_2} \<s_{i_2}|\hat{T}|s_{i_2-1}\>\cdots \<s_1|\hat{T}|s_M\> \nn\\
&= \frac{1}{Z_1}\Tr(\hat{T}^{M-i_1} \hat{\s}^z \hat{T}^{i_1 - i_2}\s^z \hat T^{i_2}) \qquad (i_1 > i_2).
\label{eq:correlatorastrace}
\ee
Here, we introduced the Pauli spin operator $\s^z$ that measures the spin of a state
\be
\s^z|s\> &= s|s\>.
\ee
It is easy to compute the correlation function (\ref{eq:correlatorastrace}) by expressing $\hat s$ in the eigenbasis of $\hat T$.
\begin{exercise}
Show that in the limit of large $M$ and large ``distance" $i_1-i_2$, the correlator factorizes into a product of expectation values $\<0|\s^z|0\>$, plus exponential corrections from the excited state
\be
\<s_{i_1} s_{i_2}\>_{\Z_M} &= \<0|\s^z|0\>^2 + O(e^{-m_\mathrm{gap} \tau}, e^{- m_\mathrm{gap}(L-\tau)}),
\ee
where $\tau\equiv (i_1-i_2)\Delta\tau$, $L\equiv M\Delta\tau$.
\end{exercise}

Let us write (\ref{eq:correlatorastrace}) in a slightly different way by introducing ``Heisenberg picture" operators
\be
\hat s(i) &= \hat T^{-i} \s^z \hat T^i.
\ee
Equation~(\ref{eq:correlatorastrace}) is equivalent to
\be
\label{eq:heisenbergtwopt}
\<s_{i_1} s_{i_2}\>_{\Z_M} &= \frac 1 {Z_1} \Tr(\hat s(i_1) \hat s(i_2) \hat T^M) \qquad (i_1 > i_2).
\ee
Note that in deriving (\ref{eq:correlatorastrace}, \ref{eq:heisenbergtwopt}), we used that $i_1>i_2$. If instead $i_2 > i_1$, the path integral would give a product of operators in the opposite order
\be
\<s_{i_1} s_{i_2}\>_{\Z_M} &= \frac 1 {Z_1} \Tr(\hat s(i_2) \hat s(i_1) \hat T^M ) \qquad (i_2 > i_1).
\ee
The general statement is that the path integral becomes a time-ordered product of quantum operators\footnote{We hope that the time-ordering symbol $T\{\cdots\}$ will not be confused with the transfer matrix.}
\be
\label{eq:periodiccorrelationlattice}
\<s_{i_1} \cdots s_{i_n}\>_{\Z_M} &= \frac{1}{Z_1}\Tr(T\{ \hat s(i_1) \cdots \hat s(i_n)\}\hat T^M ),
\ee
where the definition of the time-ordering symbol is
\be
T\{\hat s (i_1)\cdots \hat s(i_n)\} &\equiv \hat s(i_1)\cdots \hat s(i_n) \theta(i_1 > \cdots > i_n) + \mathrm{permutations}.
\ee
Here $\theta(i_1 > \cdots > i_n)$ is $1$ if the $i_k$ are in the specified order and zero otherwise.

Finally, in the limit of large-$M$, the factor $\frac{\hat T^M}{\l_+^M}$ projects onto the ground state, so a path integral on the full lattice $\Z$ becomes a vacuum expectation value of a time-ordered product
\be
\label{eq:fulllattice}
\<s_{i_1}\cdots s_{i_n}\>_\Z &= \<0|T\{ \hat s(i_1) \cdots \hat s(i_n)\}|0\>.
\ee

\begin{exercise}
Consider adding a next-to-nearest neighbor interaction to the 1d Ising model
\be
S[s] &= \sum_{i} K s_{i}s_{i+1} + K' s_{i} s_{i+2}.
\ee
Find the associated quantum Hilbert space and transfer matrix, and compute the partition function on $\Z_M$ in the limit of large $M$.
\end{exercise}

\subsection{Quantization in quantum mechanics}

The procedure of turning a path integral into a product of quantum operators is called quantization. It is an extremely general procedure that ultimately stems from ``cutting and gluing" rules of the path integral.

For a more familiar example, let us review quantization for the path integral of a quantum particle on the real line. This theory has Euclidean action
\be
S[x] &= \int d\tau \p{\frac 1 2 \dot x^2 + V(x)},
\ee
where $x(\tau)$ is a map from Euclidean time $\tau$ to $\R$.
%To quantize, we introduce an orthonormal basis state $|x\>$ for every classical configuration at a fixed time $\tau_0$. These are the analogs of $|s=\pm 1\>$ in the Ising case, which were in one-to-one correspondence with classical configurations at a fixed lattice site $i_0$.
Consider the path integral on the interval $\tau\in[\tau_a,\tau_b]$ with fixed boundary conditions
\be
\label{eq:qmtransitionamplitude}
U(x_b,\tau_b;x_a,\tau_a) &\equiv \int_{\substack{x(\tau_b)=x_b \\ x(\tau_a) = x_a}} Dx(\tau) e^{-S[x]}.
\ee
By grouping paths according to their positions at a fixed time $\tau_c\in(\tau_a,\tau_b)$, we obtain a simple ``gluing" rule for $U$,
\be
\label{eq:gluingruleqm}
U(x_b,\tau_b;x_a,\tau_a) &= \int_{-\oo}^\oo dx_c U(x_b,\tau_b;x_c,\tau_c)U(x_c,\tau_c;x_a,\tau_a).
\ee

To quantize the theory, we build up the path integral  in small time-increments using the gluing rule. Suppose we have already computed the path integral $U(x_0,\tau_0;x_a,\tau_a)$ on the interval $[\tau_a,\tau_0]$. To extend $\tau_0\to \tau_0+\e$, we write
\be
\label{eq:gluingforsmallinterval}
U(x_1,\tau_0+\e;x_a,\tau_a) &= \int dx_0\, U(x',\tau_0+\e;x_0,\tau_0) U(x_0,\tau_0;x_a,\tau_a).
\ee
The quantity
\be
U(x_1,\tau_0+\e;x_0,\tau_0) &= U(x_1,\e;x_0,0).
\ee
plays the role of a transfer matrix, and $U(x_0,\tau_0;x_a,\tau_a)$ plays the role of a state at time $\tau_0$. To see this, we introduce a Hilbert space
\be
\label{eq:qmhilberspace}
\mathcal{H} &= \mathrm{Span}\{|x\> : x\in \R\}
\ee
with inner product $\<x'|x\>=\de(x-x')$.
The $|x\>$ are analogs of the basis states $|s=\pm 1\>$ in the Ising model. Defining $\hat T_\e$ and $|\Psi(\tau_0)\>$ by
\be
\<x_1|\hat T_\e|x_0\> &= U(x_1,\e;x_0,0),\nn\\
\<x_0|\Psi(\tau_0)\> &= U(x_0,\tau_0;x_a,\tau_a),
\ee
equation (\ref{eq:gluingforsmallinterval}) becomes
\be
\label{eq:transfermatrixforqmeq}
|\Psi(\tau_0+\e)\> &= \hat T_\e |\Psi(\tau_0)\>.
\ee

To recover the Schrodinger equation, we must show that 
\be
\hat T_\e &= 1 - \e \hat H + O(\e^2),
\ee
where
\be
\hat H &= -\frac 1 2 \frac{\ptl^2}{\ptl x^2} + V(x)
\ee
is the usual quantum-mechanical Hamiltonian. This is a standard argument, and we give a quick version here for completeness. We have
\be
\<x_1|\hat T_\e|x_0\> &= \int_{\substack{x(\e)=x_1 \\ x(0) = x_0}} Dx(\tau) e^{-S[x]},\qquad(\e \ll 1)\nn\\
S[x] &= \int_0^\e d\tau \p{\frac 1 2 \dot x^2 + V(x)}.
\ee
Because the time interval is so short, if $|x(\tau)-x_0|$ ever becomes larger than $O(\e)$, the kinetic term will cause the amplitude to be highly suppressed. Thus, let us assume $|x(\tau)-x_0|$ is of order $\e$. This means we can replace $V(x)\to V(x_0)$ (up to subleading corrections in $\e$), so the potential factors out of the integrand
\be
e^{-S[x]} &= e^{-\e V(x_0)} \exp\p{\int_0^\e d\tau \frac{\dot x^2}{2}}.
\ee

We can now split $x(\tau)$ into a classical term and a fluctuation term $\z(\tau)$ with boundary conditions $\z(0)=\z(\e)=0$,
\be
x(\tau) &= x_0\p{1-\frac{\tau}{\e}} + x_1\frac{\tau}{\e} + \z(\tau),\nn\\
\int_0^\e d\tau \frac{\dot x^2}{2} &= \frac{(x_1-x_0)}{2\e} + \int_0^\e d\tau \frac{\dot\z^2}{2}.
\ee
The path integral over $\z$ contributes a constant $A_\e$ that depends on how the theory is regulated.\footnote{A very simple regulator is to approximate $x(\tau)$ as a piecewise linear path with segments of length $\e$. This is equivalent to simply setting $\z=0$ and not doing the integral.} We thus find
\be
\<x_1|\hat T|x_0\> &= A_\e e^{-\e V(x_0)} e^{-\frac{(x_1-x_0)^2}{2\e}}(1+O(\e)).
\ee
Finally, the Gaussian factor can be expanded in $\e$ using
\be
\label{eq:simplifyexponential}
e^{-\frac{x^2}{2\e}} &= \sqrt{2\pi \e}\p{\de(x) + \frac \e 2 \de''(x) + O(\e^2)},
\ee
which gives
\be
\<x_1|\hat T|x_0\> &= A_\e \sqrt{2\pi \e} \p{\de(x_1-x_0) - \e\p{-\frac 1 2\de''(x_1-x_0) + V(x_0)\de(x_1-x_0)} + O(\e^2)} \nn\\
 &= A_\e \sqrt{2\pi \e}\<x_1|1 - \e \hat H + O(\e^2)|x_0\>.
\ee
The prefactor $A_\e \sqrt{2\pi \e}$ is an overall regulator-dependent constant that can be renormalized away by adding a ``cosmological constant" term to the action.

Thus, in the limit $\e\to 0$, equation (\ref{eq:transfermatrixforqmeq}) becomes a Euclidean version of the Schrodinger equation
\be
\frac{d}{d\tau} |\Psi(\tau)\> &= - \hat H|\Psi(\tau)\>.
\ee
Integrating using the initial condition $U(x',0,x,0)=\de(x-x')$, we find
\be
\label{eq:pathintegralwithbcsasamplitude}
U(x_b,\tau_b;x_a,\tau_a) &= \<x_b|e^{-\hat H (\tau_b-\tau_a)} |x_b\>.
\ee

Path integrals on different geometries correspond to different quantum mechanical observables. For example, the path integral on a circle $S^1_\b$ of length $\b$ with periodic boundary conditions is equal to
\be
\int_{-\oo}^\oo dx\, U(x,\b;x,0) = \Tr(e^{-\b \hat H}),
\ee
which is the quantum partition function at inverse temperature $\b$.
This is the continuum analog of equation~(\ref{eq:partitionfunctionastransfermatrix}).
Inserting observables into the path integral gives a time-ordered product of the associated quantum operators. For example, with periodic boundary conditions,
\be
\label{eq:periodiccorrelation}
\<x(\tau_1) \cdots x(\tau_n)\>_{S^1_\b} &=
\int_{x(\b)=x(0)} Dx\, x(\tau_1) \cdots x(\tau_n) e^{-S[x]}\nn\\
 &= \Tr(T\{\hat x(\tau_1) \cdots \hat x(\tau_n)\} e^{-\b \hat H}),
\ee
where we introduced the operator
\be
\hat x(\tau) &= e^{\tau \hat H} \hat x e^{-\tau \hat H}, \nn\\
\hat x|x\> &= x|x\>.
\ee
To derive (\ref{eq:periodiccorrelation}), we cut the interval $[0,\b]$ into segments $[0,\tau_1]\cup [\tau_1,\tau_2] \cup \cdots [\tau_n,\b]$ (assuming $\tau_n > \cdots > \tau_1$) and use (\ref{eq:pathintegralwithbcsasamplitude}) for each segment.
The operator $\hat x$ is the analog of $\hat s$ and equation~(\ref{eq:periodiccorrelation}) is the analog of (\ref{eq:periodiccorrelationlattice}).

Finally, in the limit $\b\to \oo$, the factor $e^{-\b \hat H}$ projects onto the ground state. Thus the Euclidean path integral on $\R$ is equal to a time-ordered correlation function in the ground state,
\be
\<x(\tau_1) \cdots x(\tau_n)\>_{\R} &= \<0|T\{\hat x(\tau_1) \cdots \hat x(\tau_n)\}|0\>.
\ee
This is the continuum analog of (\ref{eq:fulllattice}).

\stoplecture

\subsection{The 2d Ising model}

Let us now consider a slightly more complicated case: the 2d Ising model. This will be a good toy example for how cutting, gluing, and quantization work in a general QFT. For simplicity, we set $h=0$. We consider the partition function on the doubly-periodic lattice $\Z_{M} \x \Z_{N}$ and label spins $s_{i,j}$ by a pair $(i,j) \in \Z_{M} \x \Z_{N}$.

The action is given by
\be
S[s] &= -K\sum_{i,j} (s_{i,j}s_{i+1,j}  +  s_{i,j}s_{i,j+1}) \nn\\
&= K \sum_{i,j} \p{\frac 1 2 (s_{i,j+1} - s_{i,j})^2 - 1} - K \sum_{i,j}  s_{i,j} s_{i+1,j} \nn\\
&= \mathrm{const.} + \sum_{j=1}^{N} L( \bs_{j+1}, \bs_{j}).
\label{eq:unimportant}
\ee
In the last line, we split the action into contributions from pairs of neighboring rows.
The notation $s_{j}$ represents the configuration of spins in the $j$-th row,
\be
(\bs_{j})_{i} &= s_{i,j}.
\ee
The action associated with a pair of neighboring rows is given by
\be
L(\bs',\bs) &= \frac 1 2 K \sum_{i=1}^{M}(\bs'_i - \bs_i)^2 - \frac 1 2 K \sum_{i=1}^{M}(\bs_{i+1} \bs_i + \bs'_{i+1} \bs'_i).
\ee
(The constant in (\ref{eq:unimportant}) gives an unimportant multiplicative constant in the path integral that will disappear in normalized correlation functions, so we will ignore it.)

To quantize the theory, we can think of the $j$ direction as time, so $\bs_{j}$ is interpreted as a classical configuration on a fixed time-slice. The Hilbert space has an orthonormal basis vector for each such configuration,
\be
\label{eq:twodisinghilbert}
\mathcal{H}_{M} &= \mathrm{Span}\left\{|\pm\! 1, \pm 1, \cdots,\pm 1\>\right\} \nn\\
&= \bigotimes_{i=1}^{M} \mathcal{H}_i,
\ee
where $\mathcal{H}_i$ is a 1-qubit Hilbert space for each site $i$. $\mathcal{H}_{M}$ is the quantum Hilbert space of $M$ qubits, and is $2^{M}$-dimensional.

The transfer matrix between successive time slices is a $2^{M} \x 2^{M}$ matrix with entries
\be
\<\bs'|\hat T|\bs\> &= e^{-L(\bs',\bs)}.
\ee
The partition function on $\Z_{M} \x \Z_{N}$ is then
\be
\label{eq:partitionfunctionastracetwod}
Z(\Z_{M} \x \Z_{N}) &= \Tr_{\mathcal{H}_{M}}(\hat T^{N}).
\ee
To compute correlation functions, we need an operator that measures the spin at site $i$. This is simply the Pauli spin matrix $\s^z_i$ associated with the $i$-th site
\be
\s^z_i |s_1,\dots,s_j,\dots,s_{M}\> &= s_i |s_1,\dots,s_i,\dots,s_{M}\>.
\ee
We also have Heisenberg picture operators
\be
\label{eq:oldspinop}
\hat s_{i,j} &= \hat T^{-j} \s_i^z \hat T^j.
\ee
Correlation functions become traces of time-ordered products, e.g.
\be
\<s_{i_1,j_1} s_{i_2,j_2}\> &= \Tr_{\mathcal{H}_{M}}(\hat T^{N+j_2-j_1} \s_{i_1}^z \hat T_2^{j_1-j_2} \s_{i_2}^z) \th (j_1-j_2) + (1\leftrightarrow 2) \nn\\
&= \Tr_{\mathcal{H}_M}(T\{\hat s_{i_1,j_1} \hat s_{i_2,j_2}\} \hat T^{N}).
\ee

Let us now emphasize an important new ingredient in the 2-dimensional case compared to the 1-dimensional case. To arrive at (\ref{eq:partitionfunctionastracetwod}), we had to choose $j$ as the time direction. We then cut the path integral along rows of constant $j$. However, we could just as well have chosen $i$ as the time direction and cut the path integral along columns of constant $i$. This would give a different Hilbert space $\mathcal{H}_N$ with dimension $2^N$, a new transfer matrix $\hat T'$ (acting on $\mathcal{H}_N$), and a different formula for the same path integral,
\be
Z(\Z_M\x\Z_N) &= \Tr_{\mathcal{H}_N}(\hat T'^M) = \Tr_{\mathcal{H}_M}(\hat T^N).
\ee
In this new quantization, an insertion of $s_{i,j}$ in the path integral becomes an insertion of
\be
\label{eq:newspinop}
s_{i,j} &\to \hat T'^{-i} \s_j^z \hat T'^i
\ee
in a time-ordered product of quantum operators. Let us emphasize that the operators (\ref{eq:oldspinop}) and (\ref{eq:newspinop}) truly are different, even though they represent the same path integral variable. They even act on different-dimensional Hilbert spaces ($2^M$ vs.\ $2^N$)! Furthermore, even the notion of ``time"-ordering is different in this different quantization: in one case, operators are ordered according to their $i$ coordinates, and in the other they are ordered according to their $j$ coordinates.

\begin{exercise}
Show how to quantize the 2d Ising model in yet another way, using $i+j$ as the time coordinate and $i-j$ as space coordinate. What is the Hilbert space? What is the transfer matrix?
\end{exercise}

%More generally, we can build up the 2d Ising path integral by cutting and gluing path integrals on arbitrary regions on the lattice. For example, given a subset of spins $B\subset \Z^2$, let the ``boundary" $\ptl B$ consist of the spins in $B$ that have nontrivial interactions with spins outside of $B$. We use $s_b$ to indicate spins in the boundary $\ptl B$ and $s_i$ to indicate spins in the interior. We can associate a Hilbert space to $\ptl B$ by introducing an orthonormal basis vector for every spin configuration on $\ptl B$,
%\be
%\mathcal{H}_{\ptl B} &= \mathrm{Span}\left\{|s_b\> : \textrm{$s_b$ is a configuration of spins on $\ptl B$} \right\}.
%\ee
%The path integral over the interior with fixed boundary conditions gives a state $|\Psi\> \in \mathcal{H}$ with wavefunction
%\be
%\<s_{b 0}|\Psi\> &= \sum_{\substack {s_i=\pm 1 \\ s_b = s_{b0}}} e^{-S[s]}.
%\ee
%Consider now the complement $\bar B$, which also has $\ptl B$ as a boundary, and let $|\Psi'\>\in \mathcal{H}$ be the state produced by performing the path integral over the interior of $\bar B$. The path integral over $\bar B\cup B$ is given by taking a product of the associated wavefunctions and summing over boundary spins 
%\be
%Z(\bar B\cup B) &= \sum_{s_b} \<\Psi'|s_b\>\<s_{b}|\Psi\> = \<\Psi'|\Psi\>.
%\ee
%
%In general when two regions $B_1,B_2$ share part of their boundary, we can partially glue states in $\mathcal{H}_{\ptl B_1}$ and $\mathcal{H}_{\ptl B_2}$ by summing over the shared boundary spins. This gives a state in the Hilbert space $\mathcal{H}_{\ptl(B_1\cup B_2)}$ (figure~\ref{}).
%
%In the special case where we cut the path integral into pieces related by a translation symmetry, we can introduce a transfer matrix and interpret correlators in terms of time-ordered products of quantum operators.

\subsection{Atiyah-Segal axioms}
\label{sec:asaxioms}

We are now ready to understand some axioms of continuum QFT. We will simultaneously state them and give examples in a Lagrangian theory of a real scalar field $\f$. A version of these axioms in TQFTs is due to Atiyah and a version in 2d CFTs is due to Segal. We will be interested in QFTs that depend on the geometry of the space in which they live, so when we say ``manifold" below, we mean ``Riemannian manifold with metric."

Consider a $d$-dimensional QFT $Q$. 
\begin{enumerate}
\item To every $(d-1)$-manifold $N$ without boundary, $Q$ assigns a Hilbert space $\mathcal{H}_{N}$ called the space of states on $N$.

For example, in theories with an explicit path integral over some fields (``Lagrangian theories"), $\mathcal{H}_N$ has an orthonormal basis state for every field configuration on $N$. In scalar field theory,
\be
\label{eq:scalarfieldhilbertspace}
\mathcal{H}_N &= \mathrm{Span}\{|\f_b\> : \f_b\in C(N,\R)\}.
\ee
with inner product $\<\f_{b1}|\f_{b2}\> = \de(\f_{b1}-\f_{b2})$ (a functional $\de$-function).

Here $C(N,\R)$ is a space of real-valued functions on $N$.\footnote{We are being deliberately vague about {\it which\/} space of functions on $N$, e.g.\ $C^1(N)$, $C^\oo(N)$, etc.. In reality, the path integral is defined by introducing a regulator and taking a limit as the regulator is removed. Different regulators will result in different spaces of functions, and we prefer to be agnostic about which regulator is being used.}  More formally, we can write
\be
\cH_N &= L^2(C(N,\R),\C),
\ee
where $L^2(X)$ denotes the square-integrable complex functions on $X$.

The space $\cH_N$ is the analog of the Hilbert spaces (\ref{eq:qmhilberspace}) and (\ref{eq:twodisinghilbert}) for quantum mechanics and the 2d Ising model. Let us discuss the example of quantum mechanics more explicitly. Quantum mechanics is a $1$-dimensional QFT. In this case $d-1=0$, and the only possible $0$-manifolds are disjoint unions of points. The space of field configurations on a point is $C(\bullet,\R)=\R$, so the Hilbert space associated to a point is $\cH_\bullet = L^2(C(\bullet,\R),\C) = L^2(\R,\C)$. This is spanned by the usual basis states $|x\>$ with inner product $\de(x-x')$.\footnote{Note in particular that $L^2(C(\bullet,\R),\C)$ is totally different from $L^2(\bullet,\C)=\C$. In general, don't confuse the Hilbert space $\cH_N$ with the space of square-integrable functions $L^2(N,\C)$.}

%Note that $L^2(C(N,\R),\C)$ totally different from $L^2(N,\C)$. To understand this, consider the special case of quantum mechanics, which is a $1$-dimensional QFT. The only possible $(d-1)$-manifolds are then disjoint unions of points. The space of fields on a point is $C(\bullet,\R)=\R$, so the Hilbert space is $\cH_\bullet = L^2(C(\bullet,\R)) = L^2(\R,\C)$. This is spanned by the usual basis states $|x\>$ with inner product $\de(x-x')$. This is not the same as $L^2(\bullet,\C)=\C$.

\item The space of states on a disconnected union is a tensor product
\be
\label{eq:disjointunionrule}
\mathcal{H}_{N_1 \sqcup N_2} &= \mathcal{H}_{N_1}\otimes \mathcal{H}_{N_2}.
\ee
This should be clear in scalar theory from the definition (\ref{eq:scalarfieldhilbertspace}). It will be useful to assign a Hilbert space to the empty manifold, and the only reasonable choice consistent with (\ref{eq:disjointunionrule}) is
\be
\mathcal{H}_{\varnothing} &= \mathbb{C}.
\ee

\item To every $d$-manifold $M$ with incoming boundary $N$ and outgoing boundary $N'$, $Q$ assigns a transition amplitude\footnote{By incoming vs. outgoing boundary, we assume that each boundary component $N$ comes equipped with a nonzero infinitesimal vector field pointing into or out of $M$.}
\be
\label{eq:transitionamplitude}
Z_M : \mathcal{H}_N \to \mathcal{H}_{N'}.
\ee
In Lagrangian theories, $Z_M$ is obtained by fixing boundary conditions on $N,N'$ and performing the path integral over the interior. For example in a scalar theory, the matrix elements of $Z_M$ are given by
\be
\label{eq:transitionmatrixelement}
\<\f_b'|Z_M|\f_{b}\> &= \int_{\substack{\phi|_{N} = \f_b \\ \phi|_{N'} = \f_b'}} D\phi(x) e^{-S[\f]}.
\ee
Here, the notation $\f|_{N}$ means the restriction of $\f$ to $N$.

In the special case where $M$ is a closed manifold (without boundary), $Z_M$ is a linear map
\be
\label{eq:closedmanifold}
Z_M : \mathbb{C} \to \mathbb{C}.
\ee
A linear map $\C\to\C$ is simply multiplication by complex number. That number is the ``partition function" of the QFT on $M$.

Another important case is where $M$ has boundaries $(\varnothing,N)$. (The first element of the ordered pair is the incoming boundary and the second is the outgoing boundary.) In this case, $Z_M$ gives a linear map
\be
\label{eq:outgoingboundary}
Z_M: \C \to \mathcal{H}_N.
\ee
Any such map is uniquely determined by its action on $1\in \C$, so this is equivalent to a state $|\Psi\>=Z_M(1) \in \mathcal{H}_N$. Thus, we can ``prepare a state" by performing the path integral with fixed boundary conditions. We saw several examples in the previous sections, for example the definition of $Z_\mathrm{partial}(i,s)$ in the 1d Ising model.

Finally, consider the case where $M$ has boundaries $(N,\varnothing)$. Then we have a linear map
\be
\label{eq:incomingboundary}
Z_M: \mathcal{H}_N \to \C,
\ee
which is equivalent to a bra-state $\<\Psi|\in \mathcal{H}_N^*$.

The usefulness of assigning $\mathcal{H}_\varnothing=\C$ is that all these special cases (\ref{eq:closedmanifold}, \ref{eq:outgoingboundary}, \ref{eq:incomingboundary}) are covered by one axiom (\ref{eq:transitionamplitude}).

\item Suppose $M_1$ has boundaries $(N,N')$ and $M_2$ has boundaries $(N',N'')$.  We can form a new $d$-manifold $M_1\cup_{N'} M_2$ by gluing $M_1$ and $M_2$ along $N'$. This manifold has boundaries $(N,N'')$. The total transition amplitude is the composition
\be
\label{eq:compositionrule}
Z_{M_1\cup_{N'} M_2} &= Z_{M_2} Z_{M_1} : \mathcal{H}_N \to \mathcal{H}_{N''}.
\ee

In scalar field theory, this follows by cutting the path integral along $N'$ by fixing boundary conditions $\f_b'\in C(N',\R)$, and then finally performing the $(d-1)$-dimensional path integral over $\f_b'$,
\be
&\int_{\substack{\phi|_{N} = \f_b \\ \phi|_{N''} = \f_b''}} D\phi(x) e^{-S[\f]} \nn\\
&= \int_{y\in N'} D\f_b'(y)
\int_{\substack{\phi_2|_{N'} = \f_b' \\ \phi_1|_{N''} = \f_b'' \\ x \in M_2}} D\phi_2(x) e^{-S[\f_2]}
\int_{\substack{\phi_1|_{N} = \f_b \\ \phi_1|_{N'} = \f_b' \\ x \in M_1}} D\phi_1(x) e^{-S[\f_1]}.
\ee
This is the field-theory analog of our gluing rule for transition amplitudes in quantum mechanics (\ref{eq:gluingruleqm}).
In the notation (\ref{eq:transitionmatrixelement}), we have simply inserted a complete set of states on $N'$,
\be
\<\f_b''|Z_{M_2} Z_{M_1}|\f_b\> &= \int_{y\in N'} D\f'_b(y) \<\f_b''|Z_{M_2}|\f_b'\>\<\f_b'|Z_{M_1}|\f_b\>.
\ee

\item \draftnote{Describe generalization to correlation functions.}

\item \draftnote{Discuss orientation, orientation reversal, conjugation.}

\end{enumerate}

\draftnote{Discuss the possibility of giving the manifolds different amounts of structure, for instance topological structure, a Riemannian metric, etc. In these notes, we will initially be interested in manifolds with a metric. Later, we will see that in certain cases we can relax this structure to only a conformal class of metrics.}

Although we have given explicit prescriptions in the case of a continuum theory, all of these axioms have analogs in the case of the 2d Ising model. It is not hard to imagine how the continuum axioms might arise in the continuum limit of the Ising model, at least in the case where the ``manifolds" $M,N,\dots$ are regions of flat space.

\stoplecture

\subsection{Quantization in continuum QFT}

Let us use the above axioms to describe the quantization procedure one more time, now in continuum QFT.
Consider the $d$-manifold $M_L=I_{L} \x N$, where $I_L=[0,L]$ is an interval of length $L$ and $N$ is a $(d-1)$-manifold, and suppose $M_L$ is endowed with the product metric. The composition rule (\ref{eq:compositionrule}) implies that
\be
\label{eq:exponentiatedtransition}
Z_{M_L} &= e^{-L \hat H}
\ee
for some operator
\be
\hat H:\mathcal{H}_N \to \mathcal{H}_N,
\ee
namely the Hamiltonian.

Consider a correlation function on $S_\b^1 \x N$
\be
\label{eq:correlatorcontinuum}
\<\f(\tau_1,y_1)\cdots \f(\tau_n,y_n)\>_{S_\b^1 \x N} &= 
\int_{\f(\b,y)=\f(0,y)} D\f(x)\, \f(\tau_1,y_1)\cdots \f(\tau_n,y_n) e^{-S[\f]}.
\ee
Here, $\tau,y$ are coordinates on $S_\b^1,N$, respectively.
As before, we introduce operators $\hat \f(y)$ that measure the field configuration on a time-slice
\be
\hat \f(y)|\f_b\> &= \f_b(y)|\f_b\>.
\ee
To compute (\ref{eq:correlatorcontinuum}), we cut the path integral at times $\tau_1,\dots,\tau_n$ and repeatedly use (\ref{eq:exponentiatedtransition}). By the usual logic, this gives
\be
\<\f(\tau_1,y_1)\cdots \f(\tau_n,y_n)\>_{S_\b^1 \x N} &= \Tr_{\mathcal{H}_N}(e^{-\b\hat H} T\{\hat \f(\tau_1,y_1)\cdots \hat \f(\tau_n,y_n)\}),
\ee
where
\be
\hat \f(\tau,y) &\equiv e^{\tau \hat H} \hat \f(y) e^{-\tau \hat H},
\ee
and $T\{\cdots\}$ represents time ordering in $\tau$.
Taking the limit of a long interval, the factor $e^{-\b \hat H}$ projects onto the ground state, and we obtain\footnote{In (\ref{eq:limitoflongbeta}), we are assuming that the theory has a unique ground state on $N$. More generally, there might be multiple ground states on $N$, in which case the limit $\b\to \oo$ becomes a trace in the space of ground states.}
\be
\label{eq:limitoflongbeta}
\<\f(\tau_1,y_1)\cdots \f(\tau_n,y_n)\>_{\R \x N} &=\<0|T\{\hat \f(\tau_1,y_1)\cdots \hat \f(\tau_n,y_n)\}|0\>.
\ee

For a QFT in $\R^d$, we can quantize the theory in many different ways by choosing different directions to play the role of ``time." In every quantization of the theory, we have
\be
\<\f(x_1)\cdots \f(x_n)\>_{\R^d} &= \<0|T\{\hat \f(x_1)\cdots \hat \f(x_n)\}|0\>.
\ee
However, in different quantizations, the objects appearing on the right-hand side are different. The Hilbert spaces are different (though isomorphic if the theory is $\SO(d)$-invariant), the ground states $|0\>$ are different, the quantum operators $\hat\f(x)$ are different, and the time-ordering symbols means different things. However, because they originate from a single path integral, the resulting expectation values are the same. It is sometimes useful to think of expressions like the RHS above as ``interpretations" of one underlying path integral.

\draftnote{Discuss boundary conditions and whether we should think of $\R^d$ as having a boundary or not.}

\subsection{The quantum transverse-field Ising model}

After all this abstract nonsense, let's return to more concrete things and try to understand the critical point of the 2d Ising model. The transfer matrix $\hat T$ of the 2d Ising lattice model was diagonalized in 1944 by Onsager. Unfortunately, we won't have time to describe his solution. Instead, we will study a closely related model in the same universality class as the lattice model.

Recall that the transfer matrix has matrix elements $\<\bs'|\hat T|\bs\> = e^{-L(\bs',\bs)}$. Let us write $\hat T$ in a more familiar way as an operator on a spin chain. First split $L$ into contributions from horizontal and vertical bonds
\be
L(\bs',\bs) &= L_h(\bs')+L_h(\bs) + L_v(\bs',\bs). \nn\\
L_h(\bs) &= -\frac 1 2 K \sum_i \bs_{i+1} \bs_i, \nn\\
L_v(\bs',\bs) &= \sum_i \frac 1 2 K(\bs_i'-\bs_i)^2.
\ee
Note that
\be
e^{\frac 1 2 K \sum_i \s^z_i \s^z_{i+1}}|\bs\> &= e^{-L_h(\bs)}|\bs\>.
\ee
Meanwhile, $L_v$ only involves spins at a single site, so let us imagine that we have only one site. Note that
\be
\<s'|(1+e^{-2K}\s^x)|s\> &= e^{-\frac 1 2 K (s'-s)^2}.
\ee
We also have
\be
1+e^{-2K} \s^x &= e^{A+K' \s^x},\qquad\textrm{where} \nn\\
\tanh K' &= e^{-2K},\nn\\
e^A &= \sqrt{1-e^{-4K}},
\ee
which follows by expanding out the Taylor series for $e^{A+K'\s^x}$ and matching the coefficients of $1,\s^x$.
Thus,
\be
e^{AM}\<\bs'|e^{K' \sum_i \s^x_i}|\bs\> &= e^{-L_v(\bs',\bs)}.
\ee
The constant $e^{AM}$ will cancel in correlation functions, so we will ignore it. Putting everything together, we find
\be
\hat T &\propto \exp\p{\frac 1 2 K \sum_i \s^z_i \s^z_{i+1}} \exp\p{K' \sum_i \s^x_i} \exp\p{\frac 1 2 K \sum_i \s^z_i \s^z_{i+1}}.
\ee
In a quantum-mechanical interpretation, we would write $\hat T=e^{-\De\tau \hat H}$, but the resulting $\hat H$ would be very complicated.

To get the theory we'll actually study, we first allow $K',K$ to be independent parameters, and then take a limit where $K,K'$ become small with fixed ratio $K'/K=g$. These modifications may seem somewhat drastic, but it turns out that, at the critical point, they can be compensated by simply rescaling the time coordinate. (We will not prove this, unfortunately.) This results in
\be
\hat T &\to e^{-\Delta \tau \hat H},
\ee
where $\Delta\tau$ is small and the Hamiltonian $\hat H$ is relatively simple,
\be
\hat H &= \sum_i \s_i ^z \s_{i+1}^z + g \sum_i \s_i^x.
\ee
This is the Hamiltonian of the ``quantum transverse-field Ising model" (TFIM). It describes a 1-dimensional chain of quantum spins with a nearest-neighbor interaction in the $z$-direction and an applied ``transverse" magnetic field in the $x$-direction.\footnote{Don't confuse the transverse field in the quantum Ising model with the applied magnetic field $h$ in the thermodynamic Ising model. At the beginning of this discussion, we set $h=0$.}${}^{,}$\footnote{A similar procedure starting from the $d$-dimensional classical Ising model gives the quantum TFIM on a $(d-1)$-dimensional spatial lattice, which is again in the same universality class.}

\subsubsection{Solution via the Jordan-Wigner transformation}

In the remainder of this section, let us solve the TFIM by diagonalizing $\hat H$. Since our Hilbert space is a tensor product of two possible states for each site, it's tempting to think of it as a fermionic Fock space, with creation and annihilation operators
\be
\s^\pm_n &\equiv \frac 1 2 (\s^y_n \pm i \s_n^z).
\ee
This is not correct, since these operators commute rather than anticommute at different sites. However, this can be fixed with a classic trick called the Jordan-Wigner transformation. We define fermionic creation and annihilation operators
\be
c^\dag_n = \p{\prod_{i=1}^{n-1} \s^x_i} \s^+_n,\qquad c_n &= \p{\prod_{i=1}^{n-1} \s^x_i }\s_n^-,
\ee 
which now satisfy canonical anticommutation relations
\be
\label{eq:anticommut}
\{c^\dag_n,c_m\} = \de_{nm},\qquad \{c_n,c_m\}=\{c_n^\dag,c_m^\dag\} = 0.
\ee
These new creation and annihilation operators are {\it nonlocal} on the spin chain. We can think of them as creating and destroying fermionic solitons.

\begin{exercise}
Verify the anticommutation relations (\ref{eq:anticommut}).
\end{exercise}

It's perhaps not surprising that we can write nonlocal variables that behave like fermions. The surprise is that the Hamiltonian in these new variables is still local, and actually quite simple
\be
\label{eq:jordanwignerhamiltonian}
H &= \sum_{n=1}^{M-1} (c_n^\dag+c_n)(c_{n+1}^\dag-c_{n+1})-P(c_{M}^\dag+c_{M})(c_{1}^\dag-c_{1})+g\sum_{n=1}^{M}(2c_n^\dag c_n -1)
\ee
where $P = \prod_{n=1}^{M}\s_n^x=(-1)^F$ is the parity operator.  Within each parity eigenspace, the Hamiltonian becomes simply
\be
H &= \sum_{n=1}^{N} (c_n^\dag+c_n)(c_{n+1}^\dag-c_{n+1})+g\sum_{n=1}^{N}(2c_n^\dag c_n -1),
\ee
where when $P=-1$ we must impose periodic boundary conditions $c_{M+1}=c_1$, and when $P=1$ we must impose antiperiodic boundary conditions $c_{M+1}=-c_1$.\footnote{The choice of which eigenspace $P=\pm 1$ to consider is equivalent to a choice of spin structure. The TFIM is a bosonic theory and can be formulated without choosing a spin structure. However, once we introduce fermions, we are required to choose a spin structure.}  Our Hamiltonian is now translationally invariant and quadratic in creation and annihilation operators, so it can be diagonalized via a Fourier transform and Bogoliubov transformation:
\be
\label{eq:afterfourier}
H &= \sum_k \left[{-(2\cos k-2g)c_k^\dag c_k-i\sin k\p{c^\dag_{-k}c_k^\dag+c_{-k}c_k}}\right]-Mg\\
&= \sum_k \e(k)\p{b_k^\dag b_k-\frac 1 2}\label{eq:diagonalizedhamiltonian},
\ee
where $b_k^\dag$ and $b_k$ are new canonically normalized creation and annihilation operators, and
\be
\label{eq:dispersionrelation}
\e(k)&= 2\sqrt{g^2-2g\cos(k)+1}
\ee
is the dispersion relation for the free fermionic quasiparticles created by $b^\dag_k$.  Because of the boundary conditions imposed by parity, the quasimomenta must take the values
\be
\label{eq:circularchainquantization}
k=
\begin{cases}
\frac{2m\pi}{M} & P=-1,\\
\frac{(2m+1)\pi}{M} & P=1.
\end{cases}
\ee
for $m=0,1,\dots,M-1$.

\draftnote{Write out the Bogoliubov transformation in more detail.}

\begin{exercise}
Derive expressions (\ref{eq:jordanwignerhamiltonian}), (\ref{eq:afterfourier}), and (\ref{eq:diagonalizedhamiltonian}).
\end{exercise}

In the continuum limit $M\to \oo$, $k$ becomes continuous. Correlation functions are dominated by the smallest values of $\e(k)$, which (assuming $g>0$) occur near $k=0$. Expanding around this point, we have
\be
\e(k)^2 &= 4(1-g)^2+ 4g k^2 + O(k^4).
\ee
Thus, the mass gap is given by $m_\mathrm{gap}=|g-1|$. At the special point $g=g_c=1$, the mass gap goes to zero and we have a critical point. Long-distance correlation functions are dominated by states with arbitrarily low energy, which requires $k\to 0$. Fur such states, we can drop the $O(k^4)$ term, and we obtain a relativistic dispersion relation\footnote{To obtain the usual dispersion relation $\e=|k|$, we need to redefine either time or space by a factor of 2.}
\be
\e(k)^2 &= 4 k^2 \qquad(g=g_c,\ k\ll 1).
\ee

To summarize, in the continuum limit, at the critical coupling, and at long distances, the quantum TFIM becomes a free relativistic fermion. This is yet another example of IR equivalence/duality. The IR theory has an emergent $\SO(1,1)$ Lorentz symmetry that was not present in the original spin system. In fact, it is also has emergent conformal symmetry, as we'll see later.

The fact that the critical point of the 1+1d quantum TFIM (and consequently other theories in the same universality class, like the 2d Ising lattice model) is equivalent to a free theory is very special to 2-dimensions. Our derivation relied in a fundamental way on the Jordan-Wigner transformation and does not work, for example in 2+1d.\footnote{There are newly-discovered versions of the Jordan-Wigner transformation in higher dimensions (\arXiv{1711.00515} --- one of your classmates is a coauthor!), but they do not relate the 2+1d TFIM to a free theory.}

\stoplecture

\section{CFTs in perturbation theory}

\subsection[Scaling and renormalization]{Scaling and renormalization\footnote{Sources for this section: Pufu \cite{Pufu2017}, Cardy \cite{Cardy:1996xt}.}}

In the quantum TFIM, we found $m_\mathrm{gap}=|g-g_c|$, which means that the correlation length critical exponent $\nu$ (defined in equation~(\ref{eq:correlationlengthexponent})) is $\nu=1$ for this theory. In this section, we explain how critical exponents are related to operator dimensions in the scale-invariant QFT (SFT) that describes the critical point.

Consider a microscopic system that has a long-distance description in terms of a quantum field theory QFT$_\mathrm{IR}$. We will think of QFT$_\mathrm{IR}$ in the language of effective field theory: it has a length cutoff $a$ and a set of coupling constants $g_A$ determined by running and matching. 
\begin{enumerate}
\item[] {\bf Matching:} First, we choose a cutoff $a\sim a_\mathrm{UV}$ near the characteristic length scales of the microscopic theory. We fix the coupling constants $g_A(a)$ at this scale by computing physical observables in both the UV and IR theories at distances $x\gtrsim a$ and demanding that they agree term-by-term in a power series in $a/x$. To match more terms, we must adjust more coupling constants.
\\
\\
An important point is that this matching procedure is analytic in the UV parameters. Specifically, if the action of the UV theory has a parameter $t$, then $g_A(a)$ will generically have a regular power-series expansion in $t$
\be
g_A(a) &= \g_{A0} + \g_{A1} t + \g_{A2} t^2 + \dots.
\ee
Roughly speaking this is because the matching happens at a fixed length scale, while nonanlyticities come from summing up contributions over many scales, as we will see shortly.\footnote{Put another way, the matching computation depends on degrees of freedom at some high energy scale. These degrees of freedom don't care about whether the theory will eventually reach a fixed-point at low energies (that is a low-energy question). From the point of view of the UV degrees of freedom, the point $t=0$ is just some generic point in the space of couplings and there is no reason to have a singularity there.}
\\
\item[] {\bf Running:} Now we perform a ``coarse-graining" operation where we rescale the cutoff $a\to ba$ with $b=1+\de \ell$, $\de\ell\ll 1$ and adjust the coupling constants $g_A$ to leave the long-distance physics invariant.\footnote{In practice, we usually use a modified version of this condition, where we keep the physics invariant up to corrections of the form $(a/x)^n$ for some power $n$ that depends on the desired accuracy. This goes hand-in-hand with the matching procedure: first we decide on how many power-law corrections $(a/x)^n$ we want to keep, we then match coupling constants up to this order and find RG equations that preserve observables up to this order. A common simplification is to throw away all corrections $(a/x)^n$ with $n>0$ --- i.e.\ take the limit as the cutoff is completely removed $a\to 0$. This is the version of RG you find in most textbooks.} The adjustment takes the form
\be
\label{eq:betafun}
a\frac{d g_A}{d a} &= - \b_A(g),
\ee
where $\b_A(g)$ is a vector-field in the space of couplings, called the ``beta function." The trajectories defined by (\ref{eq:betafun}) are ``renormalization group (RG) flows."\footnote{The beta function $\b_A$ is analytic in $g_A$ by a similar argument to the one above: it is computed by comparing two nearby length/energy scales.}

\end{enumerate}

An example RG flow is plotted in figure~\ref{}. We will be interested in fixed-points where $\b_A(g_*)=0$. At a fixed-point, observables become scale-invariant. To see this explicitly, note that by dimensional analysis, any dimensionless observable (for example a dimensionless ratio of correlation functions) must have the form
\be
G\p{\frac{x_{ij}}{a};g_A},
\ee
where $x_{ij}$ are various distances and $g_A$ are coupling constants. The $\b$-function is defined by demanding that physical observables are invariant under simultaneously rescaling $a$ and modifying the $g_A$, i.e.\
\be
\p{a\pdr{}{a} + \b_A(g) \pdr{}{g_A}} G\p{\frac{x_{ij}}{a};g_A} &= 0.
\ee
Evaluating this equation at a fixed point $g_*$, we find
\be
\label{eq:scalingform}
a\pdr{}{a} G\p{\frac{x_{ij}}{a};g_*} = 0  \qquad\implies \qquad G\p{\frac{x_{ij}}{a};g_*} &= f\p{\frac{x_{ij}}{x_{kl}}},
\ee
i.e.\ $G$ is a function of ratios of distances alone.

A fixed-point of the RG flow must be a critical point. The reason is that the scaling behavior (\ref{eq:scalingform}) is inconsistent with the long-distance behavior
\be
\<\cO(x) \cO(y)\> &\sim e^{-m_\mathrm{gap}|x-y|} \qquad (|x-y|\gg a).
\ee
The only exceptions are $m_\mathrm{gap}=0$ (so that we have a nontrivial scale-invariant theory) or $m_\mathrm{gap}=\oo$ (so that we have a TQFT).

Very close to a fixed point, $\b_A$ takes the form
\be
\b_A &= - \sum_B Y_{AB} u_B + O(u^2),
\ee
where $u=g-g_*$. Redefining the $u_A$ to diagonalize $Y_{AB}$,\footnote{It is possible to have situations where $Y_{AB}$ cannot be diagonalized, but rather has a nontrivial Jordan block form. This occurs, for example in logarithmic CFTs, which we may discuss briefly later. However, logarithmic CFTs are not unitary. We will prove later that in unitary CFTs, the matrix $Y_{AB}$ is diagonalizable. For now, let us assume it.} we have
\be
\label{eq:linearapprox}
\b_A \approx - y_A u_A,
\ee
which integrates to
\be
u_A \sim b^{y_A}.
\ee
We call the $u_A$ ``scaling variables."

\subsubsection{Relevant, marginal, and irrelevant variables}

From here, we can distinguish three cases
\begin{itemize}
\item $y_A > 0$: the coupling $u_A$ is relevant and grows at long distances. An RG flow in the $u_A$-direction takes us away from the fixed-point.
\item $y_A = 0$: the coupling $u_A$ is marginal. The fate of an RG flow in the $u_A$-direction depends on higher-order terms in the $\b$-function.
\item $y_A < 0$: the coupling $u_A$ is irrelevant. An RG flow in the $u_A$-direction takes us towards the fixed-point.
\end{itemize}

Relevant couplings must be tuned to stay close to the fixed point. From our discussion of the phase diagram of the Ising model, the fixed-point of the critical Ising model must have two relevant couplings: a thermal scaling variable $u_t$ and a magnetic scaling variable $u_h$. In addition, there are an infinite number of irrelevant variables $u_3,\dots$.\footnote{When relevant variables respect a symmetry but irrelevant variables do not, that symmetry will be emergent at long-distances. The irrelevant variables that are initially nonzero depend on the microscopic theory. For example, in magnets with no applied magnetic field, all $\Z_2$-odd irrelevant variables will vanish by symmetry. However, in liquids the $\Z_2$-symmetry is emergent, so there will in general be nonzero $\Z_2$-odd irrelevant variables. In systems where rotational invariance is emergent, there will also be irrelevant variables that transform nontrivially under $\SO(d)$, e.g.\ a tensor $u_{\mu_1\cdots\mu_j}$.} By our discussion of matching, the scaling variables $u_t,u_h$ should be analytic in the parameters of the UV theory. They must also vanish when $t=h=0$ and be consistent with $\Z_2$ symmetry. Thus,
\be
u_t &= t/t_0 + O(t^2,h^2) \nn\\
u_h &= h/h_0 + O(th),
\ee
where $t_0$ and $h_0$ are non-universal constants.\footnote{Generically, the map between UV and IR parameters is non-degenerate (i.e. differentiable with invertible derivative) at the fixed-point, so e.g. $t_0,h_0$ are not $0$ or $\oo$. To get a nondegenerate map, we might need to find the right parameters in the microscopic theory to wiggle. For example, if we had a magnet such that $u_t \sim t^2$ (I don't know of any case where this actually happens), then in addition to dialing the temperature and magnetic field, we could also compress the magnet with some pressure $P$. The three-dimensional map between the UV parameters $(T,H,P)$ and scaling variables $u_t,u_h,u_3$ will then generically be nondegenerate. If it still isn't, we can add an additional parameter, etc..}
 

In the action, scaling variables $u_A$ multiply local operators $\Phi_A$ that transform in a simple way under coarse-graining. The action linearized around the fixed-point is given by
\be
S &= S_0 + \int \frac{d^d x}{a^d} \sum_A u_A \Phi_A(x).
\ee
Here, factors of the length-cutoff $a^{-d}$ are needed to make $S$ dimensionless. For example, if the UV theory is a lattice model, $a^{-d}$ comes from writing the sum over lattice sites as an integral over space
\be
\sum_{i\in \Z^d} \to \int \frac{d^d x}{a^d}.
\ee
In our conventions, the $\Phi_A$ have classical dimension zero. (The word ``classical" is to distinguish from another notion of dimension that we will introduce shortly.)

In order for the action to be invariant under coarse-graining, we must change the operators according to
\be
\label{eq:coarsegrainingtransf}
a &\to b a, \nn\\
u_A &\to b^{y_A} u_A, \nn\\
\Phi_A &\to b^{\De_A} \Phi_A,
\ee
where
\be
y_A + \De_A = d.
\ee
The number $\De_A$ is called the ``scaling dimension" of $\Phi_A$.

At the fixed-point $u_A=0$, our coarse-graining rules fix the two-point function
\be
\label{eq:coarsegrainingtwopt}
\<\Phi_A(x_1) \Phi_A(x_2)\> &\propto \frac{a^{2\De_A}}{|x_1-x_2|^{2\De_A}}.
\ee
The right hand side is the only quantity consistent with Poincare invariance, with having classical dimension zero, and with the transformations (\ref{eq:coarsegrainingtransf}).

We can simplify this argument by introducing a version of dimensional analysis specialized to our fixed-point. We define operators and coupling constants with nontrivial length dimension by 
\be
\label{eq:redefineops}
\cO_A &= a^{-\De_A} \Phi_A, \nn\\
\l_A &= a^{-y_A} u_A = a^{\De_A - d} u_A.
\ee
These combinations are invariant under the transformation (\ref{eq:coarsegrainingtransf}). Invariance under coarse-graining then means that correlators of $\cO_A$'s can be functions of $\l_A$ (and kinematic variables like distances), but are independent of $a$. The tradeoff is that correlators must now also be consistent with dimensional analysis under which $\cO_A$ has length-dimension $-\De_A$ and $\l_A$ has length-dimension $-y_A = \De_A-d$.

For example, at the fixed-point $\l_A=0$, a two-point function of $\cO_A$ must be given by
\be
\label{eq:twoptfunctionscaling}
\<\cO_A(x_1) \cO_A(x_2)\> &\propto \frac{1}{|x_1-x_2|^{2\De_A}}.
\ee
The right-hand side is the only quantity consistent with Poincare symmetry and dimensional analysis. (Of course, it also follows trivially from (\ref{eq:coarsegrainingtwopt}) and (\ref{eq:redefineops}).)\footnote{Note that (\ref{eq:twoptfunctionscaling}) determines the critical exponent
$\eta = 2\De_h +2 - d$.}

We can now analyze the action using this specially-adapted version of dimensional analysis. We have
\be
\label{eq:deformedaction}
S &= S_0 + \int d^d x \sum_{A} \l_A \cO_A(x).
\ee
Let us split the sum into relevant and irrelevant contributions,
\be
S &= \p{S_0 + \int d^d x \sum_{\De_A < d} \l_A \cO_A(x)} + \int d^d x \sum_{\De_A > d} \l_A \cO_A(x).
\ee
(For simplicity we assume there are no marginal operators.)
The irrelevant interactions scale towards the fixed-point, so at low energies we will assume they can be treated as small perturbations of the action in parentheses.\footnote{This assumption is not always correct --- it can happen that the effects of an irrelevant interaction initially decrease near the fixed-point, but begin to grow when the effect of relevant interactions gets strong. This is called a ``dangerously-irrelevant" operator.}

For concreteness, consider the Ising model. Let us set the irrelevant $\l$'s to zero and tune $\l_h=0$ as well. There is now exactly one nonzero RG-invariant dimensionful quantity, namely $\l_t$. We can associate a mass scale to it given by
\be
m_t &= |\l_t|^{1/y_t} = |u_t|^{1/y_t} a^{-1}.
\ee
By dimensional analysis, any other dimensionful quantity must be proportional to $m_t$ to the appropriate power. (Remember that we cannot use the UV cutoff anymore because it is not RG-invariant.) For example, the mass gap of the theory in the IR must be
\be
\label{eq:massgapeq}
m_\mathrm{gap} &\propto m_t \propto |u_t|^{1/y_t} \propto |t|^{1/y_t}.
\ee
This determines the critical exponent\footnote{Our calculation of $\nu$ in the 2d Ising model shows that $\De_t=1$ for that theory. The corresponding $\cO_t$ is the mass of a 2d free fermion, which indeed has dimension $1$.}
\be
\nu=\frac{1}{y_t} = \frac{1}{d-\De_t}.
\ee


Note that the constant of proportionality in (\ref{eq:massgapeq}) can depend on the sign of $t$. The reason is that, depending on the sign of $t$, the theory may flow to different theories in the IR (figure~\ref{}). For example, in the Ising model, when $t>0$ the theory flows to a gapped phase where $\Z_2$ is unbroken, while when $t<0$, the theory flows to a gapped phase that spontaneously breaks $\Z_2$. No matter where the theory flows, $m_t$ is still the only mass-scale around, so $m_\mathrm{gap}$ must be proportional to it. However, the actual dynamics that determines $m_\mathrm{gap}$ can be arbitrarily complicated and generally depends in a nontrivial way on the actual IR theory. 

\stoplecture

\subsubsection{The cosmological constant}

An important observable is the free energy density, defined by
\be
f &= - \lim_{V\to \oo}\frac{1}{V} \log Z_{M_V},
\ee
where $Z_{M_V}$ is the partition function in a space of volume $V$ (for example a $d$-torus with side lengths $V^{1/d}$). 
Understanding $f$ involves an extra subtlety that we glossed over in the previous discussion.

An additional operator that is always present in our effective field theory is the unit operator $\mathbf 1$.  This corresponds to a ``cosmological constant" term in the action
\be
S &\supset \int d^d x\, \l_\mathbf{1}.
\ee
The coefficient $\l_\mathbf{1}$ must be matched to the UV theory just like all the others. It encodes contributions to the free energy from UV degrees of freedom not included in the IR effective field theory.

Strictly speaking $\l_\mathbf{1}$ is another relevant coupling. However, we have ignored it so far because its only physical effect is to shift the free energy density --- it cancels out of all correlation functions.

For concreteness, consider again the Ising model with no irrelevant couplings and with $\l_h$ set to zero. The only mass scale besides $\l_\mathbf{1}$ is $m_t$. Thus, by dimensional analysis, the free-energy density must take the form
\be
\label{eq:freeenergydensity}
f &= \l_\mathbf{1} + A_{\mathrm{sign}(t)} m_t^{d},
\ee
where $A_{\mathrm{sign}(t)}$ is a constant depending only on the sign of $t$ (i.e.\ which IR theory we flow to). We cannot have a more complicated combination of $\l_\mathbf{1}$ and $\l_t$ because shifting $\l_\mathbf{1}$ can only shift $f$. Thus, only the {\it singular\/} part of the free energy density displays homogeneous scaling behavior,
\be
\label{eq:singularpart}
f_\mathrm{singular} & = A_{\mathrm{sign}(t)} m_t^{d} \propto |t|^{d/y_t}.
\ee

\subsubsection{Other critical exponents}

We are now ready to compute more critical exponents.
\begin{itemize}
\item $\a$: The value of $\a$ follows immediately from (\ref{eq:freeenergydensity}, \ref{eq:singularpart}):
\be
C \propto \frac{\ptl^2 f}{\ptl t^2} \propto |t|^{d/y_t - 2} = |t|^{-\a}.
\ee
Thus,
\be
\a = 2-\frac d {y_t} = \frac{d-2 \De_t}{d-\De_t}.
\ee

\item $\b$: The magnetization is proportional to $\pdr{f}{h}\propto \pdr{f}{\l_h}$. By dimensional analysis, this must be
\be
M \propto \pdr{f}{\l_h} \propto m_t^{d-y_h} \propto |t|^\frac{d-y_h}{y_t} = |t|^\b.
\ee
Thus,
\be
\b &= \frac{d-y_h}{y_t} = \frac{\De_h}{d-\De_t}.
\ee
This is a clear case where the constant of proportionality depends on the sign of $t$. When $t$ is positive, the theory is in the $\Z_2$-unbroken phase, so the constant of proportionality is zero. When $t$ is negative, $\Z_2$ is broken and the constant of proportionality is nonzero.

\item $\gamma$: The zero-field susceptibility involves an additional derivative with respect to $\l_h$,
\be
\chi \propto \pdr{{}^2f}{\l_h^2} \propto m_t^{d-2y_h} \propto |t|^{\frac{d-2y_h}{y_t}} = |t|^{-\g}.
\ee
Thus
\be
\g &= \frac{2y_h - d}{y_t} = \frac{d-2\De_h}{d-\De_t}.
\ee

\item $\de$: To compute the magnetization at $T=T_c$, we must consider the case $\l_t=0$ and $\l_h\neq 0$. Once again, there is only one mass scale given by
\be
m_h &= |\l_h|^{1/y_h} \propto |h|^{1/y_h}.
\ee
This time, dimensional analysis gives
\be
\pdr{f}{\l_h} \propto m_h^{d-y_h} \propto |h|^\frac{d-y_h}{y_h} = |h|^{1/\de}.
\ee
Thus
\be
\de &= \frac{y_h}{d-y_h} = \frac{d-\De_h}{\De_h}.
\ee

\end{itemize}

Because all of these exponents depend only on two quantities $\De_h,\De_t$ (or $y_h,y_t$), they satisfy various ``scaling relations." For example, 
\be
\a + 2\b + \g &= 2\nn\\
\a + \b(1+\de) &= 2.
\ee
Historically, these scaling relations played an important role in verifying the above picture of the renormalization group. For us, the important point is that scaling behavior follows from the underlying assumption of having a fixed-point of the renormalization group, and the various critical exponents are determined in terms of the more fundamental operator dimensions $\De_A$.

\subsection[The Wilson-Fisher theory and the $\e$-expansion]{The Wilson-Fisher theory and the $\e$-expansion\footnote{Sources for this section: Pufu \cite{Pufu2017}.}}

As we discussed in section~\ref{}, the 3d Ising model arises as the long-distance limit of $\f^4$-theory in 3-dimensions (suitably tuned to reach a critical point). Unfortunately the corresponding fixed-point is strongly-coupled, so it is difficult to explore the above ideas or do calculations.

However, we can obtain a related, weakly-coupled fixed-point by instead considering $\f^4$-theory in $(4-\e)$-dimensions with $\e\ll 1$. This is the Wilson-Fisher (WF) theory. Observables in the WF theory can be calculated in a systematic perturbative expansion in $\e$. In principle, the 3d Ising model is obtained by resumming the $\e$-expansion and taking $\e\to 1$.

There are several issues with this idea, such as
\begin{itemize}
\item Does it even make sense to have a QFT in non-integer-dimensions?
\item Does it matter that the theory is necessarily non-unitary if $d$ is not an integer (\arXiv{})?
\item The perturbative expansion in $\e$ is generally only asymptotic. Can one compute all the nonperturbative corrections needed to go from an asymptotic expansion in $\e$ to an actual function of $\e$?
\end{itemize}
(I'm sure you can think of more.) However, none of these issues will prevent us from doing perturbation theory. Furthermore, amazingly, if we simply set $\e=1$ in the resulting perturbative series, truncated to low orders, we find results that aren't terrible! For these reasons, the $\e$-expansion has been one of the most popular tools for studying the 3d Ising model, even though it is on conceptually shaky ground. For us, it is useful because it illustrates many of the ideas of scaling and renormalization in a perturbative setting.

So let us study $\f^4$-theory in $d$-dimensions. The theory has length-cutoff $a$ and action
\be
\label{eq:WFaction}
S &= \int \frac{d^d x}{a^d} \p{\frac 1 2 a^2 \ptl_\mu \f \ptl^\mu \f + g_1 \f + \frac 1 2 g_2 \f^2 + \frac 1 {4!} g_4 \f^4 }.
\ee
(A cubic term can be removed by a field-redefinition $\f\to \f+c$.)
Here, we are departing from the usual textbook conventions and assuming that $\f,g_i$ are classically dimensionless. The textbook conventions will follow by analyzing RG transformations near the Gaussian fixed point $g_i=0$. However, there will also be another fixed-point and our starting point is more democratic between the two.

To find scaling dimensions at the Gaussian fixed-point, we should consider infinitesimal $g_i$, so that loop effects are highly suppressed. Thus, we can treat the interactions classically and simply find a coarse-graining transformation that leaves (\ref{eq:WFaction}) invariant. This is\footnote{It may look unfamiliar that coupling constants like $g_2$ depend on RG scale. This is because they differ from the usual textbook conventions by explicit powers of $a$. We can define an RG-invariant dimensionful coupling by $\l_2^\mathrm{Gaussian} = a^{-2} g_2$. This is what is usually denoted as $m^2$ in textbooks.}
\be
a &\to ba, \nn\\
\f &\to b^{\frac {d-2}{2}} \f, \nn\\
g_1 &\to b^{\frac{d}{2}+1} g_1, \nn\\
g_2 &\to b^2 g_2, \nn\\
g_4 &\to b^{4-d} g_4.
\ee
It follows that
\be
y_1 &= \frac{d}{2} + 1  \ \ > 0\qquad(\textrm{relevant}) \nn\\
y_2 &= 2\ \ \ \ \  \ \ \ > 0 \qquad(\textrm{relevant}) \nn\\
y_4 &= 4-d \begin{cases}
>0 & (d<4)\quad(\textrm{relevant}) \\
=0 & (d=4)\quad(\textrm{marginal}) \\
<0 & (d>4)\quad(\textrm{irrelevant}).
\end{cases}
\ee
Thus, when $d<4$, the quartic coupling is relevant and causes the theory to flow away from the Gaussian fixed-point. When $d=4$, the quartic coupling is marginal. A more careful calculation (done in any QFT textbook) shows that in $d=4$, $g_4$ flows logarithmically to zero in the infrared (we say $g_4$ is ``marginally irrelevant"). Finally, when $d>4$, the quartic coupling is irrelevant and the theory flows back to the Gaussian fixed-point in the IR. It turns out that the Ising model in $d>4$ is not described by the Gaussian fixed-point, so we will focus on the case $d<4$.

To understand what happens further from the Gaussian fixed-point, we need the $\b$-function. To quadratic order in the coupling-constants, this is given by
\be
\label{eq:betafns}
a \frac{dg_1}{da} &= -\b_1(g) = \frac{d+2}{2} g_1 - \frac{g_1 g_2}{d-2} + \dots \nn\\
a \frac{dg_2}{da} &= -\b_2(g) = 2g_2 - S_d g_1^2 - \frac{g_2^2}{d-2} - \frac{g_2 g_4}{2(d-2)^2 S_d} - \frac {g_4^2} {6(d-2)^3 S_d^2} + \dots \nn\\
a \frac{dg_4}{da} &= -\b_4(g) = (4-d) g_4 - 3 S_d g_2^2 - \frac{4 g_2 g_4}{d-2} - \frac{3 g_4^2}{2(d-2)^2 S_d} + \dots,
\ee
where
\be
S_d &= \frac{2\pi^{d/2}}{\Gamma(d/2)}
\ee
is the volume of the unit sphere in $d$-dimensions.\footnote{$\b$-functions themselves are not physical observables --- they depend on a choice of {\it scheme\/}, which consists of a regulator and a definition of coupling constants. The $\b$-function here comes from using conformal perturbation theory with a cutoff regulator. We will derive it later in the course. The general formula in this scheme is
\be
a\frac{ dg_A}{da} &= (d-\De_A)g_A - \frac{S_d}{2} \sum_{B,C}f_{BCA} g_{B} g_{A},
\ee
where $f_{BCA}$ is the OPE coefficient in $\cO_B \cO_C \sim f_{BCA} \cO_A$ and $\De_A$ is the scaling dimension of $\cO_A$ at the fixed-point we are perturbing around. Here we are doing conformal perturbation theory around the Gaussian fixed-point.
}
Let us make some observations. Firstly, note that $g_1=g_2=g_4=0$ is indeed a fixed-point of these equations. The $\b$-functions above are good close to this fixed-point. Note also that the linear terms follow from scaling at the Gaussian fixed-point, i.e.\ the linear term in $-\b_i$ is $y_i g_i$.

When $\e=4-d$ is small, the linear term in $\b_4$ is small, and it's possible for it to be balanced against the quadratic terms while still staying within the regime of validity of perturbation theory. This gives another fixed-point:
\be
g_{1*}=0,\quad g_{2*} = 0,\quad g_{4*} = \frac{16\pi^2}{3} \e + O(\e^2)
\ee
This is the fixed-point of the Wilson-Fisher theory.
The above approximation for $g_*$ is well-controlled because the fixed-point occurs at small values of the couplings, where we can trust the $\b$-functions (\ref{eq:betafns}). An RG flow diagram is shown in figure~\ref{}.

Linearizing the $\b$-functions around the Wilson-Fisher point, we find
\be
a\frac{dg_1}{da} &= \p{3-\frac{\e}{2}} g_1 + \dots \nn\\
a\frac{dg_2}{da} &= \p{2-\frac{\e}{3}} g_2 + \dots \nn\\
a \frac{dg_4}{da} &= -\e\p{g_4-g_{4*}}+\dots,
\ee
where we have dropped terms of higher order in the couplings and in $\e$.
Thus, at the Wilson-Fisher point we have
\be
y_1 = 3-\frac{\e}{2}+O(\e^2),\quad y_2=2-\frac \e 3+O(\e^2),\quad y_4 = -\e+O(\e^2) \nn\\
\De_{\phi} = 1-\frac \e 2+O(\e^2),\quad \De_{\phi^2} = 2-\frac{2\e}{3} + O(\e^2),\quad \De_{\phi^4} = 4+O(\e^2).
\ee
Note in particular that while $\f^4$ is relevant at the Gaussian fixed-point, it becomes irrelevant at the Wilson-Fisher point. Thus the Wilson-Fisher point has two relevant parameters, which is consistent with experimental observations of the Ising universality class. Plugging in $\e=1$, we find $\De_\f\sim 0.5$ and $\De_{\f^2}\sim 1.33$ in 3d, which is not too far from the correct answers $\De_\f\sim 0.518, \De_{\f^2}\sim 1.412$. This is nice, but not very rigorous.

Finally, let us discuss scaling variables and dimensionful couplings in this context. Consider the flow between the Gaussian point and the Wilson-Fisher point, where $g_1=g_2=0$ and $g_4=g(a)$ is a function of $a$. The $\b$-function equation is (ignoring cubic and higher terms in $g_4$)
\be
a \frac{dg}{da} = \e g (1-g/g_*),
\ee
which has solution
\be
\label{eq:wfflowsoln}
g(a) &= \frac{1}{\frac 1 {g_*} + \p{\frac{a_0}{a}}^\e}
\ee
with some dimensionful integration constant $a_0$.

When $a$ is small, (\ref{eq:wfflowsoln}) behaves like $g(a)\sim a^{\e} = a^{y_4^\mathrm{Gauss}}$, as expected. We can define an RG-invariant dimensionful coupling associated with the Gaussian fixed-point by
\be
\l_4^\mathrm{Gauss} = \left. a^{-y_4^\mathrm{Gauss}} g(a) \right|_{a\to 0} = \frac{1}{a_0^\e}.
\ee
Thus, $|\l_4^\mathrm{Gauss}|^{-1/\e}=a_0$ represents the distance scale at which the flow starts in the UV.

When $a$ is large, the solution behaves as $g(a)-g_* \sim a^{-\e} = a^{y_4^\mathrm{WF}}$. The RG-invariant dimensionful coupling associated to the Wilson-Fisher point is then given by
\be
\l_4^\mathrm{WF} = \left. a^{-y_4^\mathrm{WF}} (g(a)-g_*)\right|_{a\to\oo} = -g_*^2 a_0^{\e}.
\ee
From the point of an IR observer looking upwards in energy from the WF point, $\l_4$ encodes the leading effect of UV physics.
Note that $\l_4^\mathrm{WF}$ is suppressed by powers of the coupling compared to $1/\l_4^\mathrm{Gauss}$. This parametric separation comes from the fact that we are considering a weakly coupled flow that moves slowly as we descend in energy. In generic strongly-coupled theories, we expect scales of relevant operators in the UV and scales of irrelevant operators in the IR to be related by $O(1)$ constants.

\stoplecture

\subsection[The $O(N)$-model and the large-$N$ expansion]{The $O(N)$-model and the large-$N$ expansion\footnote{Sources for this section: Pufu \cite{Pufu2017}.}}

The Wilson-Fisher theory is one way to replace the Ising model with a related theory that has a small parameter. Another modification that is physically interesting in its own right is a generalization of the Ising model to include $N$ scalars with an $O(N)$ symmetry. The resulting theory, called the $O(N)$-model, has action
\be
\label{eq:onmodel}
S &= \int d^d x \p{\frac 1 2 \sum_{i=1}^N \ptl_\mu \f_i\ptl^\mu \f_i + \frac 1 2 m^2 \sum_{i=1}^N \f_i^2 + u \p{\sum_{i=1}^N \f_i^2}^2}
\ee
In this lecture, we will return to textbook conventions where we associate Gaussian scaling dimensions to the coupling constants and fields. Specifically $m^2$ has mass-dimension $2$ and $u$ has mass-dimension $4-d$.

Note that the $O(1)$ model is the Ising model. As discussed in the introduction, the $O(2)$ model describes XY magnets, and the $O(3)$ model describes Heisenberg magnets (among many other systems).

Here are some properties of the $O(N)$-model:
\begin{itemize}
\item It has a Gaussian fixed-point at $u=0$, $m^2=0$. At the Gaussian fixed-point, $\f_i$ has scaling dimension $\De_{\f_i} = \frac{d-2}{2}$, and the dimensions of other operators are determined by naive dimensional analysis. For example, $\De_{\f^2} = d-2$.

\item In $d<4$, the theory has another fixed-point generalizing the Wilson-Fisher fixed-point in the case $N=1$. When $d=4-\e$ with $\e\ll 1$, this fixed-point can be studied in an expansion in $\e$ for any $N$. For example, the critical coupling occurs at
\be
u_* &= \frac{2\pi^2 \e}{N+8} + O(\e^2).
\ee

\item At large $N\gg 1$, the theory can be solved in any $d$ in an expansion in $1/N$. In particular, it can be solved directly in $d=3$. One can then try to recover information about the cases $N=1,2,3$ by extrapolating to small-$N$.  This extrapolation is sometimes successful quantitatively, but it can also help build further intuition about the spectrum of the small-$N$ theories.\footnote{Another important feature of the $O(N)$ models is that at large-$N$, they have a known holographic dual in AdS$_4$. The dual theory is due to Vasiliev and the duality was discovered by Klebanov and Polyakov \arXiv{}.}

\end{itemize}

In the rest of this lecture we will set $d=3$ and study the $O(N)$ model in an expansion in $1/N$.

The large-$N$ theory is solvable because it is in some sense close to the Gaussian theory. In particular, all $O(N)$-non-singlet operators have scaling dimensions equal to their dimensions at the Gaussian fixed point, plus corrections of $O(1/N)$. For example
\be
\De_{\f_i} &= \frac 1 2 + O(1/N).
\ee
However, $O(N)$-singlet operators can have dimensions very different form their Gaussian values. For example
\be
\De_{\f^2} = 2 + O(1/N),
\ee
whereas $\De_{\f^2}^\mathrm{Gaussian} = 1$. (Here, $\f^2=\sum_i \f_i \f_i$.)

It is instructive to see how the dimension of $\f^2$ can be different at the two fixed-points. One way is to study the two-point function
\be
F(x) &= \<\f^2(x) \f^2(0)\>,
\ee
as a function of $x$ at nonzero coupling $u$. We would like to show that
\be
\label{eq:conditionsonFx}
F(x) &\sim \frac{1}{|x|^2} \qquad (x \to 0), \nn\\
F(x) &\sim \frac{1}{|x|^4} \qquad (x \to \oo).
\ee
Here, the $x\to 0$ limit probes the UV where the theory is described by the Gaussian fixed-point, while $x\to \oo$ probes the IR where the theory is described by the nontrivial critical point.

We will compute $F(x)$ by summing Feynman diagrams, and this is much easier in momentum space. In momentum space, the conditions (\ref{eq:conditionsonFx}) become
\be
\label{eq:expectedscalingmomentum}
F(p) &\sim \frac{1}{|p|} \qquad (p\to \oo), \nn\\
F(p) & \sim |p| \qquad (p\to 0),
\ee
where
\be
F(p) &= \int d^3 x\, e^{ip\.x} F(x).
\ee
The reason we can use Feynman diagrams to describe the large-$N$ theory is that only a small subclass of diagrams will be important in the large-$N$ limit.

In general to reach the critical point, we must tune $m^2$. In the scheme we will use, the critical value is $m^2=0$, so let us set $m^2=0$ from the beginning. The propagator is
\be
\<\f_i(x) \f_j(0)\> &= \de_{ij} G(x),
\ee
where
\be
G(p) = \frac{1}{p^2} \quad\implies\quad
G(x) = \frac{1}{4\pi |x|}.
\ee

\draftnote{Draw diagrams for the following discussion}

The simplest contribution to $F(x)$ comes from a bubble diagram\footnote{An even simpler possibility is to draw two disconnected loops, where we contract the two $\f$'s in $\f^2(x)$, and separately contract the two $\f$'s in $\f^2(0)$. These diagrams are divergent, and the divergence comes from the fact that we have taken two operators on top of each other. The correct procedure is to carefully define the $\f^2(x)$ operator by subtracting divergences. For example, in this case we can define it using point-splitting as $\f^2(x) = \lim_{\de x\to 0}[\f(x)\f(x+\de x) - \<\f(x)\f(x+\de x)\>]$. Ambiguities in the regularization procedure can be removed by demanding that our operator is an eigenvector under rescaling. We will make this notion more precise in a few lectures.}
\be
\<\f^2(x)\f^2(0)\> &\supset 2N G(x)^2.
\ee
The overall factor of $N$ comes from summing over $i$ indices for the two propagators. These must be equal because of the contractions in $\f^2$, so we get a single sum over $N$. In Fourier-space, the bubble diagram becomes
\be
2N G(x)^2 = \frac{2N}{16\pi^2 |x|^2} &\to \frac{1}{2} \frac{N}{2|p|}.
\ee
Another contribution to $F(x)$ comes from a pair of bubbles connected by an interaction vertex. This gives
\be
\frac 1 2 u \p{\frac{N}{2|p|}}^2.
\ee
Similarly, a chain of $k$ bubbles contributes
\be
\frac 1 2 u^{k-1} \p{\frac{N}{2|p|}}^k.
\ee

We will be interested in a limit $N\to \oo$, with the product $uN$ held fixed. In this limit, chains of bubbles are the only diagrams that contribute. Any other diagram requires reconnecting different $O(N)$ indices in a way that lowers the resulting powers of $N$ relative to the number of vertices. Thus, in this limit, we only need to sum bubble diagrams, which form a geometric series
\be
F(p) &= \frac 1 2 \p{\frac{N}{2|p|} + u \p{\frac{N}{2|p|}}^2 + u^2 \p{\frac{N}{2|p|}}^3 + \dots} \nn\\
&= \frac 1 2 \frac{N}{2|p|} \frac{1}{1-\frac{uN}{2|p|}}.
\ee

Let us check that this satisfies the expected conditions (\ref{eq:expectedscalingmomentum}). When $p$ is large, the result is indeed the same as in the Gaussian theory. When $p$ is small, we have
\be
\label{eq:fatsmallp}
F(p) &\approx -\frac 1 2 - \frac{|p|}{u^2 N} + O(p^2) \qquad (p\textrm{ small}).
\ee
At first, this looks like the wrong result. However, note that the Fourier transform of a constant is $\de(x)$, which is zero at nonzero separation $x\neq 0$. We call $\de(x)$ a ``contact term" because it is only nonzero when the points in the two-point function coincide. In general, contact terms are not captured by a low-energy effective theory.\footnote{An exception is when they are fixed by Ward identities, as we will discuss later.} Their coefficients are determined by UV physics, and we should not expect to be able to predict them from knowledge of IR scaling-dimensions alone.

The next term in (\ref{eq:fatsmallp}) is non-analytic in $p$, so its Fourier-transform is nonzero at separated points. Indeed, it has the form we expect, so that $F(x)$ behaves like $\frac{1}{|x|^4}$ for large $x$.

For higher-order computations, we can write a set of effective Feynman rules that summarize our result. We have a $\f_i$-$\f_j$ propagator of the usual form $\frac{\de_{ij}}{p^2}$, and a dotted line that represents a geometric series of bubble diagrams that behaves like $|p|$.

\subsubsection{The Hubbard-Stratonovich trick}

The Hubbard-Stratonovich trick is a more sophisticated approach that lets us re-derive the above results and systematically proceed to higher orders in $1/N$. Let us start with the partition function
\be
\label{eq:originalpartition}
Z &= \int D\f_i e^{-\int d^3 x \p{\frac 1 2 \ptl_\mu \f\. \ptl^\mu \f + u(\f\.\f)^2}}.
\ee
We can simplify the theory by introducing an auxiliary field $\s$
\be
\label{eq:hsaction}
&= \int D\f_i D\s e^{-\int d^3 x \p{\frac 1 2 \ptl_\mu \f\. \ptl^\mu \f + \frac 1 2 i \s \f\.\f + \frac{\s^2}{16 u}}}.
\ee
The $\s$ integral is Gaussian, so it can be performed by completing the square
\be
\frac{1}{16 u} (\s + 4i u \f\.\f)^2 + u(\f\.\f)^2
\ee
and dropping the quadratic term in $\s$. This gives back the original action (\ref{eq:originalpartition}).

The equation of motion for $\s$ is
\be
\label{eq:eomforsigma}
\s &= - 4iu\, \f\.\f,
\ee
so $\s$ is like $\f\.\f$. To take the IR limit of $F(p)$, we looked at small $p$. However, we could have equivalently looked at large $u$. Note that at large-$u$, the two-point function $\<\s(x) \s(0)\>$ becomes $u$-independent (unlike $\<\f^2(x)\f^2(0)\>$), so $\s$ is a better degree of freedom for describing this limit. In fact, the large-$u$, or equivalently low-energy, limit of the action (\ref{eq:hsaction}) is extremely simple:
\be
\label{eq:iraction}
Z|_\mathrm{IR} &= \int D\f_i D\s e^{-\int d^3 x \p{\frac 1 2 \ptl_\mu \f\. \ptl^\mu \f + \frac 1 2 i \s \f\.\f}}.
\ee

We can now reinterpret the fate of the $\f^2$ operator in the long-distance limit. The field $\s$ becomes a Lagrange multiplier that naively sets $\f^2=0$, removing it from the theory. However $\s$ remains, and because of the equation of motion (\ref{eq:eomforsigma}), it takes over for $\f^2$. The scaling dimension of $\s$ at large-$N$ can be determined by dimensional analysis from the action (\ref{eq:iraction}). We obtain $\De_\f=\frac 1 2$, $\De_\s=2$.

In more precise language, the IR theory (\ref{eq:iraction}) can be analyzed as follows. Note that the integral over $\f$ is Gaussian and can be done exactly
\be
\label{eq:integrateoutscalars}
Z|_\mathrm{IR} &= \int D\s e^{-\frac{N}{2} \Tr\log(-\ptl^2 + i \s)}
\ee
The factor of $N$ in the exponent comes from the fact that we have $N$ fields and each one gives a determinant of the effective kinetic term $-\ptl^2 + i\s$. So far, the manipulations we've done have been valid at any value of $N$. For generic values of $N$, the resulting theory (\ref{eq:integrateoutscalars}) is quite complicated and its not clear how to proceed. However, at large $N$ the path integral over $\s$ can be done via a saddle-point approximation. Expanding out the action to quadratic order in $\s$, we obtain a kinetic term that gives a $\s$-propagator proportional to $|p|$. From our earlier work, we know that this propagator represents an infinite series of bubble diagrams in the original theory.

The end result of this analysis is a perturbative expansion for various observables, for example\footnote{The large-$N$ expansion for $\De_\f$ is known to order $1/N^3$ \cite{} and the expansion for $\De_\s$ is known to order $1/N^2$ \cite{}.}
\be
\De_\f &= \frac 1 2 + \frac{4}{3\pi^2 N} + O(1/N^2) \nn\\
\De_\s &= 2 - \frac{32}{3\pi^2 N} + O(1/N^2).
\ee
(In practice, these are not as good as the $\e$-expansion for estimating the scaling dimensions of the Ising model.)

\subsection[Banks-Zaks fixed-points]{Banks-Zaks fixed-points\footnote{Sources for this section: Rychkov \cite{Rychkov:2016iqz}.}}

To close this section, let us discuss some very different examples of RG fixed-points that are not related to the Ising universality class (at least not in a simple way). These fixed-points arise in 4d gauge theories with particular matter content. For concreteness, consider a 4d gauge theory with gauge group $\SU(N_c)$ and $N_f$ massless Dirac fermions transforming in the fundamental representation of $\SU(N_c)$. The Lagrangian is
\be
\cL &= -\frac 1 {4g^2} F^a_{\mu\nu} F^{a\mu\nu} + \bar\psi (i\,\slash\!\!\!\! D) \psi.
\ee
The $\b$-function is
\be
\label{eq:betafn}
\b(g) &= - \b_0 \frac{g^3}{(4\pi)^2} + \b_1 \frac{g^5}{(4\pi)^4} + O(g^7) \nn\\
\b_0 &= \frac{11}{3}N_c - \frac 2 3 N_f,\qquad \b_1 \sim O(N_c^2,N_c N_f).
\ee

This theory exhibits very different behavior, depending on $N_f$ and $N_c$. Famously, when $N_f=0$, the coupling $g$ becomes large in the IR and the theory is expected to confine (meaning that asymptotic states do not carry global $\SU(N_c)$ charge). Confinement is similarly expected to happen for nonzero $N_f$ as long as $N_f$ is not too large. By contrast, if $\b_0>0$, the coupling becomes smaller in the IR and the theory is described by free particles with global $\SU(N_c)$ charge.

However, it is also clear from the $\b$-function (\ref{eq:betafn}) that the theory can exhibit very different behavior in the IR. Let us choose $N_f,N_c\gg 1$ such that $\b_0\sim O(1)>0$. The $\b$-function takes the form
\be
\label{eq:betarewrite}
\b(g) &\propto -\frac{g^3}{(4\pi)^2} \p{1- \frac{\b_1}{\b_0} \frac{g^2}{(4\pi)^2} + \dots}.
\ee
This makes it clear that there is a zero at $g=g_*$ with $g_*^2/(4\pi)^2 = O(1/N_c^2)$. The perturbative expansion can be organized in terms of the 't Hooft coupling
\be
\l_* &= \frac{N_c g_*^2}{(4\pi)^2},
\ee
which is small in the parameter range we are studying. Thus, we are justified in ignoring higher-order terms in (\ref{eq:betarewrite}), and our calculations are under perturbative control. This fixed-point is called a ``Banks-Zaks" fixed-point after the physicists who first understood that it defines a nontrivial conformal field theory.

We performed the above analysis assuming $N_f,N_c\gg 1$ with $\b_0=O(1)$ in order to ensure perturbative control over the resulting fixed-point. However, like the Wilson-Fisher theory, the Banks-Zaks fixed-point is expected to persist at least some distance outside of the perturbative regime. In supersymmetric versions of QCD, we can actually compute nonperturbatively the allowed range of $N_f,N_c$ for a conformal fixed-point --- the so-called ``conformal window." For example, with $\mathcal{N}=1$ SUSY, the conformal window is $\frac 3 2 N_c \leq N_f \leq 3 N_c$, where $N_f$ counts the number of chiral multiplets in the fundamental representation of $\SU(N_c)$. In non-supersymmetric gauge theories, the complete conditions defining the conformal window are unknown. The upper end of the conformal window $N_f < \frac{11}{2}N_c$ follows from the requirement that the theory not be IR free. However, the lower end cannot be computed perturbatively and is a topic of heated debate.\footnote{Currently, the lattice QCD community is divided over whether the lower end of the conformal window for $N_c=3$ is $N_f=12$, $N_f=11$, or perhaps somewhat lower.}

\stoplecture

\section{The bootstrap approach}

This concludes the first part of the course. We have seen why QFT is the correct language for both quantum and statistical systems, how critical points arise in several concrete examples, and how observables like critical exponents are related to scaling dimensions of operators. We have also studied these concepts in a perturbative setting. However, our main interest will be in nonperturbative theories. To describe these we will need a new set of tools. (The new tools will also help us better understand perturbative theories.)\footnote{For the next several sections, the course will closely follow my TASI Lectures \cite{}. The text below is a lightly edited version of those notes, possibly with a few examples and explicit calculations thrown in. If you want a more compact presentation, you can refer to the original set of lectures.}

Our approach for the next part of the course is based on the idea of the {\it conformal bootstrap\/}. The conformal bootstrap philosophy is to:
\begin{enumerate}
\item[0.] focus on the CFT itself and not a specific microscopic realization,
\item[1.] determine the full consequences of symmetries,
\item[2.] impose consistency conditions,
\item[3.] combine (1) and (2) to constrain or even solve the theory.
\end{enumerate}

This strategy was first articulated by Ferrara, Gatto, and Grillo~\cite{Ferrara:1973yt} and Polyakov~\cite{Polyakov:1974gs} in the 70's.  Importantly, it only uses nonperturbative structures, and thus has a hope of working for strongly-coupled theories.  Its effectiveness for studying the Ising model will become clear during this course.  In addition, sometimes bootstrapping is the {\it only\/} known strategy for understanding the full dynamics of a theory. An example is the 6d $\mathcal{N}=(2,0)$ supersymmetric CFT describing the IR limit of a stack of M5 branes in M-theory.  This theory has no known Lagrangian description, but is amenable to bootstrap analysis \cite{Beem:2015aoa}.\footnote{At large central charge, this theory is solved by the AdS/CFT correspondence \cite{Maldacena:1997re}. Supersymmetry also lets one compute a variety of protected quantities (at any central charge).  However, the bootstrap is currently the only known tool for studying non-protected quantities at small central charge.}

A beautiful and ambitious goal of the bootstrap program is to eventually provide a fully nonperturbative  formulation of quantum field theory, removing the need for a Lagrangian. % We are not there yet, but you can help!



\section{Symmetries in quantum field theory}

The first step of the conformal bootstrap is to determine the full consequences of symmetries.  In this section, we review symmetries in quantum field theory, phrasing the discussion in language that will be useful later.  We work in Euclidean signature throughout.

\subsection{The Stress Tensor}

A local quantum field theory has a conserved stress tensor,
\be
\label{eq:conservationofT}
\ptl_\mu T^{\mu\nu}(x) &= 0 \qquad \textrm{(operator equation)}.
\ee
This holds as an ``operator equation," meaning it is true away from other operator insertions.  In the presence of other operators, (\ref{eq:conservationofT}) gets modified to include contact terms on the right-hand side,
\be
\label{eq:wardidentity}
\partial_\mu \< T^{\mu\nu}(x) \cO_1(x_1)\dots \cO_n(x_n)\> &= -\sum_i \de(x-x_i)\ptl_i^\nu\<\cO_1(x_1)\dots \cO_n(x_n)\> \nn\\
&\qquad + \frac{\ptl}{\ptl x^\rho} \p{\textrm{other contact terms}}^\rho .\nn\\
\ee
This modified conservation equation is called a ``Ward identity."\footnote{Unlike the contact term discussed in the $O(N)$-model section, contact terms in Ward identities are fixed by symmetries and do not depend on details of UV physics.}

In practice, the contact terms come about because correlators have singularities as $T^{\mu\nu}$ approaches other operators of the form\footnote{There may be other singularities present, but they don't contribute to nontrivial contact terms.}
\be
\label{eq:examplesingularity}
\<T^{\mu\nu}(x) \cO(y)\cdots\> &\sim \ptl_x^\mu \ptl_x^\nu \frac{1}{(x-y)^{d-2}} \<\cO(y) \cdots\> + \dots.
\ee
Taking the divergence $\frac{\ptl}{\ptl x^\mu}$ on the right-hand side we get
\be
\ptl_x^\nu \ptl_x^2 \frac{1}{(x-y)^{d-2}} &\propto \ptl_x^\nu \de^d(x-y).
\ee

A stress tensor is always present in a continuum QFT. However, in a lattice model, a stress tensor must be emergent (if it exists).

Let us prove the Ward identity for a Lagrangian theory with a scalar field $\f$ and action $S[\phi]$.
%A correlation function is given by the path integral
%\be
%\<\cO_1(x_1)\cdots \cO_n(x_n)\> &= \int D\f\, \cO_1(x_1)\cdots \cO_n(x_n) e^{-S[\f]}.
%\ee
The stress tensor is the Noether current for translation symmetry, so we can obtain it from the Noether procedure. We consider an infinitesimal position-dependent translation parameterized by a vector field $\e_\mu(x)$. It acts on $\f$ as
\be
\label{eq:fieldtransform}
\f(x) &\to \f(x+\e(x)) = \f(x) + \de_\e \f(x), \nn\\
\de_\e \f &= \e^\mu \ptl_\mu \f(x).
\ee
Because $S[\f]$ must be invariant when $\e$ is constant, the shift in $S[\f]$ must be the integral of $\e$ times a total derivative
\be
\de_\e S[\f] &= -\int d^d x\, \e_\nu(x) \ptl_\mu T^{\mu\nu}(x),
\ee
for some operator $T^{\mu\nu}(x)$ --- the stress tensor.

For example, consider the free scalar theory in $d$-dimensions. The action is given by
\be
S_\mathrm{free} &= \int d^d x \frac 1 2 (\ptl \f)^2.
\ee
Plugging in the transformation (\ref{eq:fieldtransform}), we find
\be
\de_\e S_\mathrm{free} &= \int d^d x \ptl_\mu(\e_\nu \ptl^\nu \f) \ptl^\mu \f \nn\\
&= \int d^d x \p{\ptl_\mu \e_\nu \ptl^\nu \f \ptl^\mu \f + \e_\nu \ptl^\nu\p{\frac 1 2 (\ptl \f)^2}} \nn\\
&= -\int d^d x \, \e_\nu \ptl_\mu \p{\ptl^\nu \f \ptl^\mu \f -\frac 1 2 (\ptl \f)^2 \de^{\mu\nu}},
\ee
where we integrated by parts to get the last line. We might be tempted to identify the quantity in parentheses as the stress tensor, but the Noether procedure is actually ambiguous: it only determines the stress tensor up to a manifestly conserved term called an ``improvement term"\footnote{We can also add more complicated improvement terms, for example $\ptl_\rho B^{\rho \mu \nu}$ where $B$ is antisymmetric in $\rho\leftrightarrow \mu$. We will remark briefly on such terms in a moment.}
\be
\label{eq:improv}
T^{\mu \nu}_\mathrm{free} &= \ptl^\nu \f \ptl^\mu \f - \frac 1 2 (\ptl \f)^2 \de^{\mu\nu}  + \underbrace{(\ptl^\mu \ptl^\nu - \de^{\mu\nu} \ptl^2) \Phi}_{\textrm{improvement term}}.
\ee
Here, $\Phi$ can be any scalar (possibly composite) operator. Later, we will impose a physical principle that will pick out a particular $\Phi$. But for now any $\Phi$ will do.

Having determined the symmetry current $T^{\mu\nu}$, we can derive the Ward identity. Consider a correlation function of local operators
\be
\<\cO_1(x_1)\cdots \cO_n(x_n)\> &= \int D\f\, \cO_1(x_1)\cdots \cO_n(x_n) e^{-S[\f]}.
\ee
We can think of $\f\to \f+\de_\e \f$ as a change of variables in the path integral, and this change of variables should leave the integral invariant. In other words
\be
0 &= \int D\f\, \de_\e\p{\cO_1(x_1)\cdots \cO_n(x_n) e^{-S[\f]}} \nn\\
 &= \sum_i \<\cO_1(x_1) \cdots \de_\e \cO_i(x_i) \cdots \cO_n(x_n)\>\nn\\
 &\qquad + \<\cO_1(x_1) \cdots \cO_n(x_n) \int d^d x\, \e_\nu(x) \ptl_\mu T^{\mu\nu}(x)\>.
 \label{eq:changeofvariables}
\ee
Now note that
\be
\de_\e(\cO_i(x_i)) &= \e_\nu(x_i) \ptl_i^\nu \cO_i(x_i) + (\ptl \e)(\dots) + (\ptl^2 \e)(\dots) + \dots,
\ee
where the first term is proportional to $\e$ and remaining terms are proportional to derivatives of $\e$ at $x_i$.\footnote{Terms depending on derivatives of $\e$ would come about if $\cO_i$ depended on derivatives of $\f$, for example.} The Ward identity (\ref{eq:wardidentity}) then follows from taking a functional derivative of (\ref{eq:changeofvariables}) with respect to $\e(x)$.


There is an alternative formalism that deals more elegantly with improvement terms and ambiguities in the Noether procedure. This is to imagine coupling our QFT to a background metric in a diffeomorphism-invariant way. (Note that without diffeomorphism invariance, the theory would depend on more than just a metric, but also a choice of local coordinates.) The Ward identity can be derived in this formalism as follows.
\begin{exercise}
Consider a QFT coupled to a background metric $g$. For concreteness, suppose correlators are given by the path integral
\be
\<\cO_1(x_1)\dots\cO_n(x_n)\>_g &= \int D\phi\,\cO_1(x_1)\dots\cO_n(x_n)\, e^{-S[g,\f]}.
\ee
A stress tensor insertion is the response to a small metric perturbation,
\be
\label{eq:definitionofstresstensor}
\<T^{\mu\nu}(x)\cO_1(x_1)\dots\cO_n(x_n)\>_g &= \frac{2}{\sqrt g}\frac{\de}{\de g_{\mu\nu}(x)}\<\cO_1(x_1)\dots\cO_n(x_n)\>_g.
\ee
Derive the Ward identity\footnote{Note that this version of the Ward identity doesn't include terms proportional to $\ptl \e$, unlike (\ref{eq:wardidentity}). This is because in defining the $\cO_i$ so that they transform like scalars under diffeomorphisms, we must include factors of $g^{\mu\nu}$. For example, $(\ptl \f)^2 = g^{\mu\nu}\ptl_\mu \f \ptl_\nu \f$. The variation of these factors cancels terms proportional to $\ptl \e$. Equivalently, the definition (\ref{eq:definitionofstresstensor}) has the effect of adding contact terms to $T_{\mu\nu}$ that simplify the Ward identity.}
\be
\label{eq:wardidentitysimpl}
\partial_\mu \< T^{\mu\nu}(x) \cO_1(x_1)\dots \cO_n(x_n)\> &= -\sum_i \de(x-x_i)\ptl_i^\nu\<\cO_1(x_1)\dots \cO_n(x_n)\>
\ee
by demanding that $S[g,\phi]$ be diffeomorphism invariant near flat space. Assume that the $\cO_i$ are defined so that they transform like scalars under diffeomorphisms. Find how to modify (\ref{eq:wardidentity}) when the $\cO_i$ have spin.
\end{exercise}


The nice thing about this formalism is that there are no ambiguities in the definition of the stress tensor: it is always the response of the theory to a metric perturbation. However, now we must make an explicit choice of how to couple our theory to a background metric. In particular, there can be new couplings in the action that aren't present in flat space. For example, the action for a free-scalar in curved space admits a new curvature-dependent mass term
\be
\label{eq:freecurvedspace}
S_\mathrm{free}[g,\f] &= \int d^d x \sqrt{g} \p{\frac 1 2 g^{\mu\nu} \ptl_\mu \f \ptl_\nu \f + \frac 1 2 \xi R \f^2}
\ee
\begin{exercise}
Using the definition $T^{\mu\nu} = -\frac{2}{\sqrt g}\frac{\de S}{\de g_{\mu\nu}}$, show that the stress tensor in the theory (\ref{eq:freecurvedspace}), evaluated in flat space, is
\be
T^{\mu\nu} &= \ptl^\mu \f \ptl^\nu \f - \frac 1 2 g^{\mu\nu} (\ptl \f)^2 - \xi (\ptl^\mu \ptl^\nu -  g^{\mu\nu}\ptl^2) \f^2.
\ee
In other words, $\xi$ determines a choice of improvement term in (\ref{eq:improv}).
\end{exercise}

A related complication that this formalism avoids is that the Noether procedure doesn't automatically give a symmetric stress tensor (it does for the free scalar, but that was luck), and one must find an improvement term that fixes this. In the background metric formalism, the stress tensor is automatically symmetric.

A disadvantage of the background metric approach is that it might not be obvious how to couple a given UV system (e.g.\ a lattice model) to a nontrivial metric. For the Ising CFT, we can take advantage of the fact that there exists a continuum field theory (i.e.\ $\f^4$ theory) in its universality class, which manifestly can be coupled to curved space. However, for more general CFTs, the assumption that they can be coupled  to a background metric is nontrivial.\footnote{In fact, some theories have local gravitational anomalies that mean they are not precisely diffeomorphism invariant. However, this is a very mild breaking of diffeomorphism invariance and doesn't affect our analysis.}

\subsection{Topological operators and symmetries}

Consider the integral of $T^{\mu\nu}$ over a closed surface $\Sigma$,\footnote{The word ``surface" usually refers to a 2-manifold, but we will abuse terminology and use it to refer to a codimension-1 manifold.}${}^,$\footnote{Our definition of $P^\nu$ differs from the usual one by a factor of $i$.  This convention is much nicer for Euclidean field theories, but it has the effect of modifying some familiar formulae, and also changing the properties of symmetry generators under Hermitian conjugation. More on this in section~\ref{sec:reflectionpositivity}.}
\be
P^\nu(\Sigma) &\equiv -\int_\Sigma dS_\mu T^{\mu\nu}(x).
\ee
The Ward identity (\ref{eq:wardidentity}) implies that a correlator of $P^\nu(\Sigma)$ with other operators is unchanged as we move $\Sigma$, as long as $\Sigma$ doesn't cross any operator insertions (figure~\ref{fig:topologicalsurfaces}).
We say that $P^\nu(\Sigma)$ is a ``topological surface operator."

\begin{figure}
\begin{center}
\includegraphics[width=0.45\textwidth]{topologicalsurfaces.jpg}
\end{center}
\caption{A surface $\Sigma$ supporting the operator $P^\mu(\Sigma)$ can be freely deformed $\Sigma\to\Sigma'$ without changing the correlation function, as long as it doesn't cross any operator insertions.
\label{fig:topologicalsurfaces} }
\end{figure}

Let $\Sigma=\ptl B$ be the boundary of a ball $B$ containing $x$ and no other insertions. Integrating (\ref{eq:wardidentity}) over $B$ gives
\be
\label{eq:integratedwardidentity}
\<P^\mu(\Sigma)\cO(x)\dots\> &= \ptl^\mu\<\cO(x)\dots\>.
\ee
In other words, surrounding $\cO(x)$ with the topological surface operator $P^\mu$ is equivalent to taking a derivative (figure~\ref{fig:surroundoperator}).

\begin{figure}
\begin{center}
\includegraphics[width=0.5\textwidth]{surroundoperator.jpg}
\end{center}
\caption{\label{fig:surroundoperator} Surrounding $\cO(x)$ with $P^\mu$ gives a derivative.}
\end{figure}

In quantum field theory, having a topological codimension-1 operator is the same as having a symmetry.\footnote{Topological operators with support on other types of manifolds give ``generalized symmetries" \cite{Gaiotto:2014kfa}.}${}^,$\footnote{More precisely, to have a symmetry we usually require the topological codimension-1 operators to be invertible. If they are group-like, this means they have a multiplicative inverse, and if they are algebra-like, they have an additive inverse.}  This may be unfamiliar language, so to connect to something more familiar, let us revisit the relation between the path integral and Hamiltonian formalisms.

%
%\subsection{Quantization}
%\label{sec:quantization}
%
%A single path integral can be interpreted in terms of different time evolutions in different Hilbert spaces.  For example, in a rotationally-invariant Euclidean theory on $\R^d$, we can choose any direction as ``time" and think of states living on slices orthogonal to the time direction (figure~\ref{fig:differentquantizations}).  We call each interpretation a ``quantization" of the theory.


%
%To specify a quantization, we foliate spacetime by slices related by an isometry $\ptl_t$. A slice has an associated Hilbert space of states.  A correlation function $\<\cO_1(x_1)\cdots\cO_n(x_2)\>$ gets interpreted as a time-ordered expectation value
%\be
%\label{eq:timeordered}
%\<\cO_1(x_1)\cdots\cO_n(x_n)\> &= \<0|T\{\widehat \cO_1(t_1,\bx_1)\cdots \widehat \cO_n(t_n,\bx_n)\}|0\>.
%\ee
%Here, the time ordering $T\{\dots\}$ is with respect to our foliation, $|0\>$ is the vacuum in the Hilbert space $\cH$ living on a spatial slice,\footnote{Other choices of initial and final state correspond to different boundary conditions for the path integral.} and $\widehat\cO_i(x):\cH\to \cH$ are quantum operators corresponding to the path integral insertions $\cO_i(x)$.
  
%A different quantization of the theory would give a completely different Hilbert space $\cH'$, a completely different time-ordering, and completely different quantum operators $\widehat \cO_i'$.  However, some equations satisfied by these new operators on this new Hilbert space would be unchanged.  For example, if we arrange the operators as shown on the right-hand side of (\ref{eq:timeordered}), we always get the correlator on the left-hand side.
%
%We demonstrate these ideas explicitly in appendix~\ref{app:latticequantization}, where we show how to (discretely) quantize the lattice Ising model in different ways.


\begin{figure}
\begin{center}
\includegraphics[width=0.65\textwidth]{differentquantizations.jpg}
\end{center}
\caption{\label{fig:differentquantizations} In a rotationally invariant Euclidean theory, we can choose any direction as time.  States live on slices orthogonal to the time direction.}
\end{figure}

Recall that the path integral can be interpreted in terms of different quantizations. To specify a quantization, we foliate spacetime by slices related by an isometry $\ptl_t$ (figure~\ref{fig:differentquantizations}). A slice has an associated Hilbert space of states.  A correlation function $\<\cO_1(x_1)\cdots\cO_n(x_2)\>$ gets interpreted as a time-ordered expectation value
\be
\label{eq:timeordered}
\<\cO_1(x_1)\cdots\cO_n(x_n)\> &= \<0|T\{\widehat \cO_1(t_1,\bx_1)\cdots \widehat \cO_n(t_n,\bx_n)\}|0\>.
\ee
Here, the time ordering $T\{\dots\}$ is with respect to our foliation, $|0\>$ is the vacuum in the Hilbert space $\cH$ living on a spatial slice,\footnote{Other choices of initial and final state correspond to different boundary conditions for the path integral.} and $\widehat\cO_i(x):\cH\to \cH$ are quantum operators corresponding to the path integral insertions $\cO_i(x)$.

\begin{figure}
\begin{center}
\includegraphics[width=0.5\textwidth]{slidingcharges.jpg}
\end{center}
\caption{\label{fig:slidingcharges} The charge $P^\mu(\Sigma_t)$ can be moved from one time to another $t\to t'$ without changing the correlation function.}
\end{figure}

Let $\Sigma_t$ be a spatial slice at time $t$ and consider the operator $P^\mu(\Sigma_t)$.  Because $P^\mu(\Sigma)$ is topological, we are free to move it forward or backward in time from one spatial slice to another as long as it doesn't cross any operator insertions (figure~\ref{fig:slidingcharges}). In fact, we  often neglect to specify the surface $\Sigma_t$ and just write $P^\mu$ (though we should keep in mind where the surface lives with respect to other operators). We call $P^\mu$ ``momentum," and the fact that it's topological is the path integral version of the statement that momentum is conserved.

Let us understand what happens when we move $P^\mu$ past an operator insertion. Consider a local operator $\cO(x)$ at time $t$. If $\Sigma_1$, $\Sigma_2$ are spatial surfaces at times $t_1<t<t_2$, then when we quantize our theory, the difference $P^\mu(\Sigma_{2})-P^\mu(\Sigma_{1})$ becomes a commutator because of time ordering,
\be
\<(P^\mu(\Sigma_{2})-P^\mu(\Sigma_{1}))\cO(x)\dots\>=\<0|T\{[\widehat P^\mu,\widehat \cO(x)]\dots\}|0\>.
\ee
(We assume that the other insertions ``$\dots$" are not between times $t_1$ and $t_2$.)
Because $P^\mu$ is topological, we can deform $\Sigma_2-\Sigma_1$ to a sphere $S$ surrounding $\cO(x)$, as in figure~\ref{fig:deformingcharges}.  Then using the Ward identity (\ref{eq:integratedwardidentity}), we find
\be
\<0|T\{[\widehat P^\mu, \widehat \cO(x)]\dots\}|0\>&=\<(P^\mu(\Sigma_2)-P^\mu(\Sigma_1))\cO(x)\dots\>\nn\\
&=\<P^\mu(S)\cO(x)\dots\>\nn\\
&= \ptl^\mu\<\cO(x)\dots\>\nn\\
&= \ptl^\mu\<0|T\{\widehat \cO(x)\dots\}|0\>,
\ee
in other words,
\be
[\widehat P^\mu, \widehat \cO(x)] &= \ptl^\mu\widehat \cO(x).
\ee

\begin{figure}
\begin{center}
\includegraphics[width=0.8\textwidth]{deformingcharges.jpg}
\end{center}
\caption{\label{fig:deformingcharges} For any charge $Q(\Sigma)$, we can deform $Q(\Sigma_2)-Q(\Sigma_1)=Q(\Sigma_2-\Sigma_1)$ to an insertion of $Q(S)$. Here, arrows indicate the orientation of the surface.}
\end{figure}

Figure~\ref{fig:deformingcharges} makes it clear that this result is independent of how we quantize our theory, since we always obtain a sphere surrounding $\cO(x)$ no matter which direction we choose as ``time."  Thus, we often write
\be
\label{commutator}
[P^\mu,\cO(x)] &= \ptl^\mu\cO(x),
\ee
without specifying a quantization.  In fact, from now on, we will no longer distinguish between path integral insertions $\cO(x)$ and quantum operators $\widehat \cO(x)$.
The expression $[Q,\cO(x)]$ can be interpreted as either an actual commutator $[\widehat Q,\widehat \cO(x)]$ in any quantization of the theory, or in path-integral language as surrounding $\cO(x)$ with a topological surface operator $Q(S)$.

Figure~\ref{fig:deformingcharges} also explains why the commutator $[Q,\cO(x)]$ of a charge $Q$ with a local operator $\cO(x)$ is local, even though $Q$ is nonlocal (it is the integral of a current). The reason is that the support of $Q$ can be deformed to an arbitrarily small sphere $S$ around $x$, so that the insertion $Q(S)\cO(x)$ only affects the path integral in an infinitesimal neighborhood of $x$.  In general, the way local operators transform under symmetry is always insensitive to IR details like spontaneous symmetry breaking or finite temperature.  This is because commutators with conserved charges can be computed at short distances.

\stoplecture

Equation (\ref{commutator}) integrates to
\be
\label{eq:integratedtranslations}
\cO(x) &= e^{x\.P}\cO(0)e^{-x\.P}.
\ee
This statement is also true in any quantization of the theory.  In path integral language, $e^{x\.P}(\Sigma)$ is another type of topological surface operator.  When we surround $\cO(0)$ with $e^{x\.P}(\Sigma)$, it becomes conjugation $e^{x\.P}(\Sigma)\cO(0)\to e^{\widehat P\. x}\widehat\cO(0)e^{- \widehat P\. x}$ in any quantization.

Consider the time-ordered correlator (\ref{eq:timeordered}) with $t_n>\cdots>t_1$.  Using (\ref{eq:integratedtranslations}), it becomes
\begin{align}
&\<\cO_1(x_1)\cdots\cO_n(x_n)\>\nonumber\\
&= \<0|e^{t_n P^0}\cO_n(0,\bx_n)e^{-t_n P^0}\cdots e^{t_1 P^0}\cO_1(0,\bx_1)e^{-t_1P^0}|0\>\nonumber\\
&=\<0|\cO_n(0,\bx_n)e^{-(t_n-t_{n-1})P^0}\cdots e^{-(t_2-t_1)P^0}\cO_1(0,\bx_1)|0\>.
\end{align}
In other words, the path integral between spatial slices separated by time $t$ computes a time evolution operator $U(t)=e^{-tP^0}$. In particular, $P^0$ is the Hamiltonian, as expected.  %In unitary theories (defined in more detail in section~\ref{sec:reflectionpositivity}), $P^0$ has a positive real spectrum, so $U(t)$ causes damping at large time separations.

\subsection{A topological surface operator on the lattice}

Let us give an example of a topological surface operator that is not the integral of a current (or the  exponential of the integral of a current). We will again see that it should be thought of as a symmetry.

Consider the lattice Ising model in $d$-dimensions with zero applied magnetic field, $h=0$. The operator $U_{\Z_2}(\Sigma)$ is defined by flipping the signs of all nearest-neighbor bonds that cross the surface $\Sigma$. More precisely, let $\Sigma$ be a closed surface that does not intersect any of the lattice points. For each pair of nearest neighbor sites $\<ij\>$, define the sign\footnote{Being more careful to allow for very wiggly surfaces, we should define $p_\Sigma(i,j)$ as $(-1)^n$, where $n$ is the number of times $\Sigma$ intersects the edge $\<ij\>$.}
\be
p_{i,j}(\Sigma) &= \begin{cases}
-1 & \textrm{if the edge from $i$ to $j$ intersects $\Sigma$,} \nn\\
+1 & \textrm{otherwise}.
\end{cases}
\ee
An insertion of $U_{\Z_2}(\Sigma)$ is defined by
\be
\label{eq:pathintegralwithUinserted}
\<U_{\Z_2}(\Sigma) \cO_1\cdots \cO_n \> &\equiv \sum_{\{s_i\}}  \cO_1\cdots \cO_n e^{K\sum_{\<ij\>} p_{i,j}(\Sigma) s_i s_j}.
\ee

Why is the operator $U_{\Z_2}(\Sigma)$ topological? Consider a deformation $\Sigma\to \Sigma'$. We can transform $U_{\Z_2}(\Sigma)$ into $U_{\Z_2}(\Sigma')$ by doing a change of variables $s_i\to -s_i$ for all spins between $\Sigma$ and $\Sigma'$ in the path integral (\ref{eq:pathintegralwithUinserted}).

When we flip the spins between $\Sigma$ and $\Sigma'$, any $\Z_2$-odd operator in that region will also flip sign. Thus, $U_{\Z_2}(\Sigma)$ can be deformed freely, but when we move it past an operator, we pick up a factor of the $\Z_2$ charge of that operator. For example, if $B$ is a ball containing an operator $\cO$ (and no other operators), we have
\be
\label{eq:ztsphere}
\<U_{\Z_2}(\ptl B) \cO \cdots\> &= q_\cO\< \cO\cdots\>,\qquad q_\cO=\pm 1.
\ee
This equation is exactly analogous to (\ref{eq:integratedwardidentity}).

Consider inserting $U_{\mathbb{Z}_2}$ on a spatial slice $\Sigma_t$ at fixed time $t$.\footnote{Since we are on a lattice, $t$ is discrete. As usual, we are free to choose any lattice direction as time.} After quantization, $U_{\mathbb{Z}_2}(\Sigma_t)$ becomes an operator $\hat U_{\mathbb{Z}_2}$ that is independent of $t$ --- in other words it is conserved --- because $U_{\mathbb{Z}_2}$ is topological. The only effect of $t$ is to determine the ordering of quantum operators. For example, assuming $t_3>t_2>t_1$, we have\footnote{When $\Sigma$ is inserted between two rows of spins, we evolve from one row to the next using either $\hat T \hat U_{\mathbb{Z}_2}$ or $ \hat U_{\mathbb{Z}_2}\hat T$, where $\hat T$ is the transfer matrix. The two choices are the same because $\hat U_{\mathbb{Z}_2}$ commutes with $\hat T$. This is one way of saying that $\hat U_{\mathbb{Z}_2}$ is conserved.}
\be
\<\cO_3(t_3) U_{\Z_2}(\Sigma_{t_2}) \cO_1(t_1)\> &= \<0|\hat \cO_3(t_3) \hat U_{\mathbb{Z}_2} \hat \cO_1(t_1)|0\>.
\ee
Here $\hat \cO_i(t_i)$ are Heisenberg-picture operators as in, e.g.\ (\ref{eq:oldspinop}). The statement (\ref{eq:ztsphere}) is equivalent to
\be
\label{eq:commutatordiscrete}
\hat U_{\mathbb{Z}_2} \hat \cO \hat U_{\mathbb{Z}_2}^{-1} &= q_\cO \hat \cO.
\ee
Concretely, the operator $\hat U_{\mathbb{Z}_2}$ is given by
\be
\hat U_{\mathbb{Z}_2} = \prod_{i=1}^M \s^x_i,\qquad \hat U_{\mathbb{Z}_2}|\bs\> = |-\bs\>.
\ee
where $\bs=(s_1,\cdots, s_M)$ are the spins on a spatial slice.

By now it should be clear that $\hat U_{\mathbb{Z}_2}$ is the generator of a $\mathbb{Z}_2$ global symmetry acting on the Hilbert space. The path integral perspective shows that we can consider $U_{\mathbb{Z}_2}$ on surfaces other than just spatial slices.% This often makes symmetries more manifest.% For example from the definition (\ref{eq:uassigmaxs}), it is not clear why (\ref{eq:commutatordiscrete}) should be a local operator. However, this is obvious from the fact that $U_{\mathbb{Z}_2}(\Sigma)$ is topological, since we can take $\Sigma$ to be a very small sphere surrounding $\cO$.

\subsubsection{Implementation in the continuum}

The operator $U_{\mathbb{Z}_2}$ can also be defined in continuum $\f^4$ theory as follows. Given a surface $\Sigma$, let us cover it with small open balls $B_i$ such that $\Sigma$ splits each ball into two pieces $B_i^\mathrm{left}\cup B_i^\mathrm{right}$. We do this so we don't have to  worry about the global structure of $\Sigma$. The action without an insertion of $U_{\mathbb{Z}_2}(\Sigma)$ is the integral of a Lagrangian density
\be
S[\phi] &= \sum_{i} \int_{B_i} d^d x \cL[\f](x),\nn\\
\cL[\phi](x) &= \frac 1 2 (\ptl \f)^2 + \dots.
\ee
Note that $S[\phi]$ favors configurations where $\f$ is slowly-varying.

To insert $U_{\mathbb{Z}_2}(\Sigma)$, we would like to instead favor configurations where $\f$ flips sign as it crosses $\Sigma$. Let 
\be
\f_i'(x) &= \begin{cases} 
\f(x) & x\in B_i^\mathrm{right} \nn\\
-\f(x) & x\in B_i^\mathrm{left}
\end{cases}.
\ee
The modified action is given by
\be
S'[\phi] &= \sum_i\int_{B_i} d^d x \cL[\f'_i](x).
\ee
Inserting $U_{\mathbb{Z}_2}(\Sigma)$ means performing the path integral with the modified action $S'$,
\be
\<U_{\mathbb{Z}_2}(\Sigma) \cO_1\cdots \cO_n\> &= \int D\f\, \cO_1\cdots \cO_n e^{-S'[\f]}.
\ee
This is the continuum version of flipping the bonds that cross $\Sigma$. 

In practice, it is difficult to work with $S'[\f]$ because it has funny $\de$-functions coming from  derivatives of $\f'$. If $\Sigma=\ptl B$ is the boundary of a region $B$, we can make a change of variables $\f\to -\f$ inside $B$ to get rid of $U_{\mathbb{Z}_2}(\Sigma)$. When we do this, any $\Z_2$-odd operator inside $B$ will pick up a sign. This gives the same result (\ref{eq:ztsphere}) that we found on the lattice.

However, if $\Sigma$ is not the boundary of a region, we cannot make a change of variables to remove $U_{\mathbb{Z}_2}(\Sigma)$. As an example, consider the theory on a torus $T^d$ with $U_{\mathbb{Z}_2}$ inserted on a spatial slice $T^{d-1}\subset T^d$. This slice is homologically nontrivial, so no change of variables will remove $U_{\mathbb{Z}_2}$. However, we can simplify the action by defining $\f$ to flip sign as we cross $\Sigma$,
\be
\f(t_\mathrm{below}) = -\f(t_\mathrm{above}).
\ee
This turns $S'$ back into $S$, but with the cost of introducing ``twisted" boundary conditions for the field $\f$
\be
\label{eq:antiperiodicity}
\f(t+\b) &= -\f(t),
\ee
where $\b$ is the length of the time circle. The path integral with twisted boundary conditions is often more straightforward to evaluate.

One effect of inserting $U_{\mathbb{Z}_2}$ on a spatial slice $T^{d-1}\subset T^d$ is that correlators of $\Z_2$-odd operators in the presence of $U_{\mathbb{Z}_2}(T^{d-1})$ become antiperiodic in the time direction. This is because to move them around the time circle, we must cross $U_{\mathbb{Z}_2}$, and doing so introduces a sign. This can also be understood in terms of twisted boundary conditions: any operator made of an odd number of $\f$'s will also satisfy the antiperiodicity condition (\ref{eq:antiperiodicity}).

\subsection{More spacetime symmetries}

Let us return to discussing symmetries built from the stress tensor. Given a conserved current $\ptl_\mu J^\mu=0$ (as an operator equation), we can always define a topological surface operator by integration.\footnote{It is an interesting question whether the converse is true. When a theory has a Lagrangian description, the Noether procedure gives a conserved current for any continuous symmetry (that is manifest in the Lagrangian).  Proving Noether's theorem without a Lagrangian is an open problem.} For $P^\nu$, the corresponding currents are $T^{\mu\nu}(x)$.  More generally, given a vector field $\e=\e^\mu(x)\ptl_\mu$, the charge
\be
Q_\e(\Sigma) &= -\int_\Sigma dS_\mu \e_\nu(x) T^{\mu\nu}(x)
\ee
will be conserved whenever
\be
0&=\ptl_\mu(\e_\nu T^{\mu\nu}) \nn\\
&=
 \ptl_\mu \e_\nu T^{\mu\nu}+\e_\nu \ptl_\mu T^{\mu\nu}\nn\\
&= \frac 1 2(\ptl_\mu \e_\nu+\ptl_\nu \e_\mu) T^{\mu\nu},
\ee
or
\be
\label{eq:killingvector}
\ptl_\mu\e_\nu+\ptl_\nu\e_\mu &= 0.
\ee
This is the Killing equation. It is equivalent to the statement that $\e$ generates an isometry of the flat metric on $\R^d$. It has solutions
\begin{align}
\label{eq:poincaregenerators}
p_\mu &= \ptl_\mu &&\textrm{(translations)},\nn\\
m_{\mu\nu} &= x_\nu\ptl_\mu - x_\mu\ptl_\nu && \textrm{(rotations)}.
\end{align}
The corresponding charges
are momentum $P_\mu=Q_{p_\mu}$ and angular momentum $M_{\mu\nu}=Q_{m_{\mu\nu}}$.

\stoplecture

\section{Conformal symmetry}

In a conformal field theory, the stress tensor satisfies an additional condition: it is traceless,
\be
T_\mu^\mu(x) &= 0 \qquad\textrm{(operator equation)}.
\ee
This is equivalent to the statement that the theory is insensitive to position-dependent rescalings of the metric $\de g_{\mu\nu}=2\omega(x) g_{\mu\nu}$ near flat space.\footnote{In curved space, there can by Weyl anomalies, which we discuss in section~\ref{}.} %Tracelessness of $T_{\mu\nu}$ is our final CFT axiom:
%\begin{definition}
%A conformal field theory (CFT) is a QFT satisfying the Atiyah-Segal axioms discussed in section~\ref{sec:asaxioms}, and possessing a conserved, symmetric, traceless stress tensor.
%\end{definition}

\subsection{Scale-invariance plus unitarity implies conformal invariance in 2d}
\label{sec:scaleimpliesconformal}

We have seen that critical points exhibit scale-invariance, but it might not be obvious why they should have a traceless stress tensor. Proving that scale-invariance implies $T_\mu^\mu=0$ for general $d$ is an open problem.\footnote{An almost-complete argument exists in 4-dimensions \cite{}. Part of the challenge is identifying the appropriate assumptions to make. There exist boring/pathological counterexamples in 3d (for example free electromagnetism), and the right formulation of the theorem should exclude these cases.} However, proving it for $d=2$ is straightforward.\footnote{This argument is from Zohar Komargodski's notes \cite{KomargodskiNotes}. See \cite{YellowPages} for a similar proof in position-space.}

Consider the two-point function of the stress tensor in momentum space (this will make it easy to study conservation). Symmetry and conservation require it to take the form
\be
\label{eq:momentumspacetwopt}
\<T_{\mu\nu}(q) T_{\rho \s}(-q)\> &= \frac 1 2 ((q_\mu q_\rho - q^2 \de_{\mu \rho})(q_\nu q_\s - q^2 \de_{\nu\s}) + \rho \leftrightarrow \s) f(q^2)\nn\\
&\qquad + (q_\mu q_\nu - q^2 \de_{\mu\nu})(q_\rho q_\s - q^2 \de_{\rho \s}) g(q^2),
\ee
where $f(q^2)$, $g(q^2)$ are general functions of $q^2$. Here, we have Fourier transformed the two operators with momenta $p,q$, and stripped off an overall momentum-conserving $\de^d(p+q)$.

The form (\ref{eq:momentumspacetwopt}) holds in any $d$. However in $d=2$, something special happens: the two structures above are linearly-dependent.
\begin{exercise}
Let $d=2$, and consider $\tl q_\mu = \e_{\mu\nu} q^\nu$, where $\e_{\mu\nu}$ is the totally antisymmetric symbol:
\be
\e_{01} = 1,\quad \e_{10}=-1,\quad \e_{00}=\e_{11}=0.
\ee
Simplify the product
\be
\tl q_{\mu} \tl q_\nu \tl q_\rho \tl q_\s
\ee
in two different ways to obtain the structures multiplying $f(q^2)$ and $g(q^2)$.
\end{exercise}

Thus, in two dimensions, we have simply
\be
\label{eq:twoptmomentum2d}
\<T_{\mu\nu}(q) T_{\rho \s}(-q)\> &=(q_\mu q_\nu - q^2 \de_{\mu\nu})(q_\rho q_\s - q^2 \de_{\rho \s}) h(q^2)
\ee
for a single function $h(q^2)$.
Now suppose that the theory is also scale-invariant. This implies that $h(q^2)$ should be a pure power of $q$,\footnote{Nontrivial scale-dependence is allowed in correlation functions if it multiplies a contact term (i.e.\ the violation of scaling is invisible at separated points). In momentum space, this means we can have terms proportional to $\log(q^2/\mu^2)$, provided they multiply an analytic function of $q$ (whose Fourier transform is a contact term). In the case of $h(q^2)$, something like $q^{-2} \log(q^2/\mu^2)$ is disallowed because $\mu\frac{d}{d\mu} q^{-2} \log(q^2/\mu^2)$ is not analytic in $q$.}
\be
h(q^2) &= \frac{b}{q^2}.
\ee
The dimension of $h(q^2)$ comes from the fact that $T_{\mu\nu}(q)$ is dimensionless (it is the $d$-dimensional Fourier transform of a quantity with dimension $d$), but we have also stripped off a dimensionful $\de^d(p+q)$ to define (\ref{eq:twoptmomentum2d}).
Taking the trace, we find
\be
\<T(q) T(-q)\> &= b q^2,
\ee
where $T\equiv T^\mu_\mu$.
In position-space, this becomes a contact term
\be
\<T(x) T(y)\> &\propto b\ptl^2 \de^2(x-y).
\ee

%In a 2d scale-invariant theory, the stress tensor is a spin-2 operator with scaling dimension $2$. The most general two-point function consistent with these properties, together with Poincare-invariance and permutation symmetry is
%\be
%\<T_{\mu\nu}(x)T_{\rho\s}(0)\> &= \frac{1}{x^8} \left(A_1 \de_{\mu\nu}\de_{\rho \s} x^4 + A_2 (\de_{\mu\rho} \de_{\nu\s}+ \de_{\mu\s}\de_{\nu\rho}) x^4\right.\nn\\
%&\qquad\quad \left.+ A_3 (\de_{\mu\nu} x_\rho x_\s + \de_{\rho\s}x_\mu x_\nu)x^2 + A_4 x_\mu x_\nu x_\rho x_\s)\right),
%\ee
%where the $A_i$'s are constants.
%Conservation then implies
%\be
%\ptl^\mu\<T_{\mu\nu}(x)T_{\rho\s}(0)\> &= -\frac{1}{x^8} \left(3(A_4+2A_3)x_\nu x_\rho x_\s + (4A_1+3A_3)\de_{\rho\s} x_\nu x^2\right.\nn\\
%&\qquad\qquad \left.+ (4A_2-A_3)(\de_{\rho\nu} x_\s + \de_{\nu\s}x_\rho)\right),
%\ee
%which means $A_1,A_2,A_3,A_4$ are determined in terms of a single constant
%\be
%A_1 = 3A,\quad A_2 = -A,\quad A_3 = -4A,\quad A_4=8A.
%\ee
%The two-point function becomes
%\be
%\<T_{\mu\nu}(x)T_{\rho\s}(0)\> &= \frac{A}{x^8} \left((3 \de_{\mu\nu}\de_{\rho\s} - \de_{\mu\rho}\de_{\nu\s} - \de_{\mu\s}\de_{\nu\rho}) x^4 \right.\nn\\
%&\qquad\quad \left.-4x^2(\de_{\mu\nu}x_\rho x_\s + \de_{\rho\s} x_\mu x_\nu) + 8x_\mu x_\nu x_\rho x_\s\right).
%\ee
%Taking the trace, we find
%\be
%\<T(x) T(0)\> &= 0,
%\ee
%where $T(x)\equiv T^\mu_\mu(x)$.

In a unitary quantum field theory, if the two point function of an operator vanishes at separated points, {\it any\/} correlator involving that operator vanishes at separated points. Roughly-speaking, unitary means the theory that can be quantized in such a way that the space of states has a positive-definite inner product and the Hamiltonian is Hermitian and bounded from below. In a Euclidean theory, a more correct term is ``reflection positivity." We will discuss unitarity/reflection positivity in more detail later.

The argument for vanishing of $T(x)$ is as follows. As we discussed in the introduction, in a unitary theory, inserting a complete set of states into a two-point function gives a sum of positive terms (see equation~(\ref{eq:completeset}))
\be
\label{eq:positivesumforttwopt}
\<T(x^0,\mathbf{ 0})T(0)\> &= \sum_\psi |\<0|T(0)|\psi\>|^2 e^{-E_\psi x^0}.
\ee
If the left-hand side vanishes, each term on the right-hand side must vanish as well: $\<0|T(0)|\psi\>=0$, in other words $T(x)$ must annihilate the vacuum. Now consider a general correlation function of $T(x)$ with other operators,
\be
\label{eq:correlatorofTwithstuff}
\<T(x) \cO_1(x_1)\cdots \cO_n(x_n)\>.
\ee
This correlator vanishes if $x$ is in the past or future of all the other $x_i$, since then $T(x)$ acts on the vacuum on either the right or left. However, correlation functions in QFT are analytic in their arguments, and the analytic continuation of zero is zero. Thus, the entire correlator (\ref{eq:correlatorofTwithstuff}) vanishes.

\subsubsection{Scale without conformal invariance}

Scale without conformal invariance is possible in nonunitary theories. Perhaps the simplest example is the theory of elasticity in 2-dimensions \cite{Riva:2005gd}. This is the theory of a vector $u_\mu$ with action
\be
\label{eq:theoryofelasticity}
S &= \int d^2 x \p{g u_{\mu\nu} u^{\mu\nu} + \frac k 2 (u_\mu{}^\mu)^2},
\ee
where $u_{\mu\nu}=\frac 1 2 (\ptl_\mu u_\nu + \ptl_\nu u_\mu)$. The field $u_\mu$ is called the ``displacement field" and $u_{\mu\nu}$ is called the ``strain tensor." The coefficient $g$ represents the shear modulus, and $k+g$ is the bulk modulus of the described material.

The action (\ref{eq:theoryofelasticity}) is quadratic in $u$, so this theory is easy to quantize. Note that $u_\mu$ describes a massless vector field without a gauge redundancy. Thus, there are negative-norm states, and the theory is nonunitary. Concretely, one can define a conserved stress tensor $T^{\mu\nu}$ that satisfies the correct Ward identities, and such that $\<T_\mu^\mu T_\nu^\nu\>=0$, but $T_\mu^\mu$ is not zero. For example, it has non-vanishing correlation functions with other operators:
\be
\<T_\mu^\mu(z) :\!\ptl u \ptl u\!:\!(0)\> &= -\frac{k+g}{2\pi^2 g(k+2g)} \frac{1}{z^4}.
\ee
(Here, we use complex coordinates $z=x+iy$ and $u=u_1-i u_2$.)

This example shows that the positive sum (\ref{eq:positivesumforttwopt}) for $\<TT\>$ is crucial for completing the argument in section~\ref{sec:scaleimpliesconformal}. The proof that scale implies conformal invariance in 4d \cite{} also uses unitarity in a fundamental way. However, though unitarity may be a sufficient condition for conformal invariance, it is not necessary. We will encounter some interesting nonunitary CFTs (like the Lee-Yang theory) later in the course.

\draftnote{Discuss virial current?}

\subsection{Symmetries from tracelessness}

Recall that the charge $Q_\e(\Sigma)$ is conserved if
\be
 \frac 1 2(\ptl_\mu \e_\nu+\ptl_\nu \e_\mu) T^{\mu\nu} &= 0.
 \ee
 When the stress tensor is traceless, this condition can be solved by demanding a less constraining version of the Killing equation (\ref{eq:killingvector}),
\be
\label{eq:conformalkillingfirst}
\ptl_\mu\e_\nu + \ptl_\nu \e_\mu = c(x)\de_{\mu\nu},
\ee
where $c(x)$ is any scalar function.  Contracting both sides with $\de^{\mu\nu}$ determines $c$ in terms of $\e$,
\be
\label{eq:conformalkilling}
\ptl_\mu\e_\nu + \ptl_\nu \e_\mu = \frac{2}{d}(\ptl \. \e)\de_{\mu\nu}.
\ee

Equation (\ref{eq:conformalkilling}) is the {\it conformal\/} Killing equation. Let us find its solutions in $\R^d$. Taking $\ptl^\nu\ptl^\mu$ of both sides, we find
\be
\p{2-\frac 2 d} \ptl^2(\ptl\.\e) = 0\quad \implies \quad \ptl^2 (\ptl \. \e)=0,
\ee
where we have assumed that $d\neq 1$. Now taking $\ptl^\nu\ptl_\rho$ and symmetrizing in $(\rho\mu)$, we find
\be
\p{1-\frac{2}{d}}\ptl_\rho\ptl_\mu(\ptl \. \e)=0 %\quad\implies\quad \ptl_\rho \ptl_\mu(\ptl\.\e) = 0\textrm{ or $d=2$}.
\label{eq:equationaboutdivergence}
\ee
For the moment, let us focus on the case $d\neq 2$, so we can conclude that $\ptl_\rho \ptl_\mu(\ptl\.\e)=0$. Taking two derivatives of the conformal Killing equation and using (\ref{eq:equationaboutdivergence}), we find
\be
\ptl_\a(\ptl_\b \ptl_\g \e_\de + \ptl_\b \ptl_\de \e_\g) &= 0,\nn\\
\ptl_\a(\ptl_\g \ptl_\b \e_\de + \ptl_\g \ptl_\de \e_\b) &= 0,\nn\\
\ptl_\a(\ptl_\de \ptl_\g \e_\b + \ptl_\de \ptl_\b \e_\g) &= 0.
\ee
These are three equations for three unknowns, and they imply
\be
\ptl_\a\ptl_\b \ptl_\g \e_\de &= 0.
\ee
Thus $\e_\de$ is at most a quadratic polynomial in $x$.

From here the analysis is straightforward. Together with $p_\mu,m_{\mu\nu}$, the space of solutions is spanned by
\begin{align}
\label{eq:extraconformalgenerators}
d &= x^\mu \ptl_\mu &&\textrm{(dilatations)},\nn\\
k_\mu &= 2x_\mu (x\.\ptl)-x^2\ptl_\mu &&\textrm{(special conformal transformations)}.
\end{align}
The corresponding symmetry charges are $D=Q_d$ and $K_\mu=Q_{k_\mu}$. Overall, there are
\be
\underbrace{d}_{p_\mu} + \underbrace{\frac{d(d-1)}{2}}_{m_{\mu\nu}} + \underbrace{1}_d + \underbrace{d}_{k_\mu} &= \frac{(d+2)(d+1)}{2}
\ee
solutions to the conformal Killing equation in $d>2$, each with an associated conserved charge.

Now let us return to the case $d=2$. It is natural to write the conformal Killing equation in terms of holomorphic and antiholomorphic coordinates $\e = \e \ptl_z + \bar \e \ptl_{\bar z}$, where $\e,\bar \e$ are initially independent functions of both $z$ and $\bar z$. In this language, the conformal Killing equation becomes
\be
\ptl_z \bar \e = 0,\quad \ptl_{\bar z} \e = 0.
\ee
(The equation involving $\ptl_z \e + \ptl_{\bar z} \bar \e$ is tautological.) A general solution is given by choosing $\e$ to be a function of $z$ alone, and $\bar \e$ to be a function of $\bar z$ alone,
\be
\e &= \e(z) \ptl_z + \bar \e(\bar z) \ptl_{\bar z}.
\ee
Note that this includes the solutions $p_\mu,m_{\mu\nu},k_\mu,d$ along with many others.
Consequently, 2d CFTs have an infinite set of conserved charges given by the integral of the stress tensor against a holomorphic or antiholomorphic vector field. We will study the implications of this infinite set of charges in section~\ref{}. For now we focus on the charges that are present in any $d$.
%\footnote{The above solutions are present in any spacetime dimension.  In two dimensions, there exists an infinite set of additional solutions to the conformal Killing equation, leading to an infinite set of additional conserved quantities \cite{Belavin:1984vu}.}


\subsection{Finite conformal transformations}

Before discussing the charges $P_\mu,M_{\mu\nu},D,K$, let us take a moment to understand the geometrical meaning of the conformal Killing vectors (\ref{eq:poincaregenerators}) and (\ref{eq:extraconformalgenerators}).  Consider an infinitesimal transformation $x^\mu\to x'^\mu=x^\mu+\e^\mu(x)$.  If $\e^\mu$ satisfies the conformal Killing equation, then
\be
\label{eq:conformalinfinitesimal}
\pdr{x'^\mu}{x^\nu} &= \de^{\mu}_\nu+\ptl_\nu\e^\mu
=\p{1+\frac 1 d (\ptl\.\e)}\p{\de^\mu_\nu + \frac 1 2\p{\ptl_\nu \e^\mu - \ptl^\mu \e_\nu}}.
\ee
The right-hand side is an infinitesimal rescaling times an infinitesimal rotation.  Exponentiating gives a coordinate transformation $x\to x'$ such that
\be
\label{eq:conformalfinite}
\pdr{x'^\mu}{x^\nu} &= \Omega(x)R^\mu{}_\nu{}(x),\qquad R^TR=I_{d\x d},
\ee
where $\Omega(x)$ and $R^\mu{}_\nu{}(x)$ are finite position-dependent rescalings and rotations.
Equivalently, the transformation $x\to x'$ rescales the metric by a scalar factor,
\be
\delta_{\mu\nu}\pdr{x'^\mu}{x^\alpha}\pdr{x'^\nu}{x^\beta} &= \Omega(x)^2\de_{\alpha\beta}.
\ee
Such transformations are called {\it conformal}. They comprise the conformal group, a finite-dimensional subgroup of the diffeomorphism group of $\R^d$.

The exponentials of $p_\mu$ and $m_{\mu\nu}$ are translations and rotations. Exponentiating $d$ gives a scale transformation $x\to \lambda x$.  
We can understand the exponential of $k_\mu$ by first considering an inversion
\be
I:x^\mu \to \frac{x^\mu}{x^2}.
\ee
$I$ is a conformal transformation, but it is not continuously connected to the identity, so it can't be obtained by exponentiating a conformal Killing vector. This means that a CFT need not have $I$ as a symmetry.
\begin{exercise}
Show that $I$ is continuously connected to a reflection $x^0\to -x^0$.  Conclude that a CFT is invariant under $I$ if and only if it is invariant under reflections.
\end{exercise}

\begin{exercise}
Show that $k_\mu = -I p_\mu I$. Conclude that $e^{a\.k}$ implements the transformation
\be
\label{eq:finitespecialconformal}
x &\to x'(x)=\frac{x^\mu-a^\mu x^2}{1-2(a\.x)+a^2 x^2}.
\ee
\end{exercise}
We can think of $k_\mu$ as a ``translation that moves infinity and fixes the origin" in the same sense that the usual translations move the origin and fix infinity, see figure~\ref{fig:translationnearinfinity}.

\begin{figure}
\begin{center}
\includegraphics[width=0.5\textwidth]{translationnearinfinity.jpg}
\end{center}
\caption{\label{fig:translationnearinfinity} $k_\mu$ is analogous to $p_\mu$, with the origin and the point at infinity swapped by an inversion.}
\end{figure}

\subsection{The conformal algebra}

The charges $Q_\e$ give a representation of the conformal algebra
\be
\label{eq:conformalalgebra}
[Q_{\e_1},Q_{\e_2}] &= Q_{-[\e_1,\e_2]},
\ee
where $[\e_1,\e_2]$ is a commutator of vector fields.\footnote{The minus sign in (\ref{eq:conformalalgebra}) comes from the fact that when charges $Q_i$ are represented by differential operators $\cD_i$, repeated action reverses the order $[Q_1,[Q_2,\cO]]=\cD_2 \cD_1 \cO$.  Alternatively, we could have introduced an extra minus sign in the $Q$'s so that $[Q,\cO]=-\cD$ and then $Q,\cD$ would have the same commutation relations.}  This is not obvious and deserves proof. In fact, it is {\it not true\/} in 2-dimensional CFTs, where the algebra of charges is a central extension of the the algebra of conformal killing vectors.

\begin{exercise}
Show that in $d\geq 3$,
\be
\label{eq:conformaltransfofT}
[Q_\e,T^{\mu\nu}] &= \e^\rho\ptl_\rho T^{\mu\nu}+(\ptl\.\e)T^{\mu\nu}-\ptl_\rho \e^\mu T^{\rho\nu}+\ptl^\nu\e_\rho T^{\rho\mu}.
\ee
Argue as follows. Assume that only the stress tensor appears on the right-hand side.\footnote{Bonus exercise: can other operators appear?} Using linearity in $\e$, dimensional analysis, and the conformal Killing equation, show that (\ref{eq:conformaltransfofT}) contains all terms that could possibly appear.\footnote{The terms on the right-hand side are local in $\e$ because we can evaluate $[Q_{\e},T^{\mu\nu}(x)]$ in an arbitrarily small neighborhood of $x$. Assuming the singularity as two $T^{\mu\nu}$'s coincide is bounded, we can then replace $\e$ by its Taylor expansion around $x$.}  Fix the relative coefficients using conservation, tracelessness, and symmetry under $\mu\leftrightarrow \nu$. Fix the overall coefficient by matching with (\ref{commutator}).
\end{exercise}

\begin{exercise}
Using (\ref{eq:conformaltransfofT}), prove the commutation relation (\ref{eq:conformalalgebra}).
\end{exercise}

\begin{exercise}
When $d=2$, it's possible to add an extra term in (\ref{eq:conformaltransfofT}) proportional to the unit operator that is consistent with dimensional analysis, conservation, and tracelessness.  Find this term (up to an overall coefficient),\footnote{The coefficient can be fixed by comparing with the OPE, see e.g. \cite{Polchinski:1998rq}. It is proportional to the central charge $c$.} and show how it modifies the commutation relations (\ref{eq:conformalalgebra}). This is the Virasoro algebra!
\end{exercise}

As usual, (\ref{eq:conformalalgebra}) is true in any quantization of the theory.  In path integral language, it tells us how to move the topological surface operators $Q_{\e}(\Sigma)$ through each other.

\begin{exercise} Show that
\be
\,[M_{\mu\nu},P_\rho] &= \de_{\nu\rho}P_\mu - \de_{\mu\rho}P_\nu,\\
\,[M_{\mu\nu},K_\rho] &= \de_{\nu\rho}K_\mu - \de_{\mu\rho}K_\nu,\\
\,[M_{\mu\nu},M_{\rho\s}] &= \de_{\nu\rho}M_{\mu\s}-\de_{\mu\rho}M_{\nu\s}+\de_{\nu\s}M_{\rho\mu}-\de_{\mu\s}M_{\rho\nu},\label{eq:mmcommutator}\\
\label{eq:dpcommutator}
\,[D,P_\mu]&=P_\mu,\\
\label{eq:dkcommutator}
\,[D,K_\mu]&=-K_\mu,\\
\,[K_\mu,P_\nu]&=2\de_{\mu\nu}D-2M_{\mu\nu},
\ee
and all other commutators vanish.
\end{exercise}
The first three commutation relations say that $M_{\mu\nu}$ generates the algebra of Euclidean rotations $\SO(d)$ and that $P_\mu,K_\mu$ transform as vectors.  The last three are more interesting.  Equations~(\ref{eq:dpcommutator}) and (\ref{eq:dkcommutator}) say that $P_\mu$ and $K_\mu$ can be thought of as raising and lowering operators for $D$. We will return to this idea shortly.

\begin{exercise} Define the generators
\be
\label{eq:conformalgeneratorssodplus11}
L_{\mu\nu}&=M_{\mu\nu},\nn\\
L_{-1,0} &= D,\nn\\
L_{0,\mu} &= \frac 1 2 (P_\mu+K_\mu),\nn\\
L_{-1,\mu}&= \frac 1 2 (P_\mu-K_\mu),
\ee
where $L_{ab}=-L_{ba}$ and $a,b\in \{-1,0,1,\dots,d\}$.  Show that $L_{ab}$ satisfy the commutation relations of $\SO(d+1,1)$.
\end{exercise}
The fact that the conformal algebra is $\SO(d+1,1)$ suggests that it might be good to think about its action in terms of $\R^{d+1,1}$ instead of $\R^d$.  This is the idea behind the ``embedding space formalism" \cite{Dirac:1936fq,Mack:1969rr,Boulware:1970ty,Ferrara:1973eg,Weinberg:2010fx,Costa:2011mg}, which provides a simple and powerful way to understand the constraints of conformal invariance. We will be more pedestrian in this course, but I recommend reading about the embedding space formalism in the lecture notes by Penedones \cite{Joao} or Rychkov \cite{Rychkov:2016iqz}.

\section{Primaries and descendants}

Now that we have our conserved charges, we can classify operators into representations of those charges.  We do this in steps. First we classify operators into Poincare representations, then scale+Poincare representations, and finally conformal representations.

\subsection{Poincare Representations}

In a rotationally-invariant QFT, local operators at the origin transform in irreducible representations of the rotation group $\SO(d)$,
\be
\label{eq:rotationatorigin}
[M_{\mu\nu},\cO^a(0)]&= (\cS_{\mu\nu}){}_b{}^a\cO^b(0),
\ee
where $\cS_{\mu\nu}$ are matrices satisfying the same algebra as $M_{\mu\nu}$, and $a,b$ are indices for the $\SO(d)$ representation of $\cO$.\footnote{The funny index contractions in (\ref{eq:rotationatorigin}) ensure that $M_{\mu\nu}$ and $\cS_{\mu\nu}$ have the same commutation relations (exercise!).}\footnote{Because our commutation relations (\ref{eq:mmcommutator}) for $\SO(d)$ differ from the usual conventions by a factor of $i$, the generators $\cS_{\mu\nu}$ will be {\it anti-}hermitian, $\cS^\dag=-\cS$.}  We often suppress spin indices and write the right-hand side as simply $\cS_{\mu\nu}\cO(0)$. The action (\ref{eq:rotationatorigin}), together with the commutation relations of the Poincare group, determines how rotations act away from the origin.

\begin{figure}
\begin{center}
\includegraphics[width=0.5\textwidth]{commutatorissurround.jpg}
\includegraphics[width=0.5\textwidth]{surroundmany.jpg}
\end{center}
\caption{The shorthand notation $Q\cO$ stands for surrounding $\cO$ with a surface operator $Q(\Sigma)$. Equivalently, it stands for $[Q,\cO]$ in any quantization of the theory. \label{fig:commutatorissurround}}
\end{figure}

To see this, it is convenient to adopt shorthand notation where commutators of charges with local operators are implicit, $[Q,\cO] \to Q \cO$, see figure~\ref{fig:commutatorissurround}.  This notation is valid because of the Jacobi identity (more formally, the fact that adjoint action gives a representation of a Lie algebra).  In path integral language, $Q_n\cdots Q_1 \cO(x)$ means surrounding $\cO(x)$ with topological surface operators where $Q_n$ is the outermost surface and $Q_1$ is the innermost.  The conformal commutation relations tell us how to re-order these surfaces.

Acting with a rotation on $\cO(x)$, we have
\be
M_{\mu\nu}\cO(x) &= M_{\mu\nu}e^{x\.P}\cO(0) \nn\\
&= e^{x\.P}(e^{-x\.P} M_{\mu\nu} e^{x\.P})\cO(0)\nn\\
&= e^{x\.P}(-x_\mu P_\nu + x_\nu P_\mu+M_{\mu\nu})\cO(0)\nn\\
&= (x_\nu \ptl_\mu - x_\mu\ptl_\nu+\cS_{\mu\nu})e^{x\.P}\cO(0)\nn\\
&= (m_{\mu\nu}+\cS_{\mu\nu})\cO(x).\label{eq:actionbyrotation}
\ee
In the third line, we've used the Poincare algebra and the Hausdorff formula
\be
e^{A}Be^{-A} = e^{[A,\.]}B
= B+[A,B]+\frac 1 {2!}[A,[A,B]]+\dots.
\ee

\subsection{Scale+Poincare representations}

In a scale-invariant theory, it's also natural to diagonalize the dilatation operator acting on operators at the origin,
\be
\label{eq:dilatationcondition}
[D,\cO(0)]&=\Delta \cO(0).
\ee
The eigenvalue $\Delta$ is the {\it dimension\/} of $\cO$.
\begin{exercise}
Mimic the computation (\ref{eq:actionbyrotation}) to derive the action of dilatation on $\cO(x)$ away from the origin,
\be
\label{eq:dilatationaction}
[D,\cO(x)] &= (x^\mu\ptl_\mu + \Delta)\cO(x).
\ee
\end{exercise}

Equation (\ref{eq:dilatationaction}) is constraining enough to fix two-point functions of scalars up to a constant.  Firstly, by rotation and translation invariance, we must have
\be
\<\cO_1(x)\cO_2(y)\>&=f(|x-y|),
\ee
for some function $f$.

In a scale-invariant theory with scale-invariant boundary conditions, the simultaneous action of $D$ on all operators in a correlator must vanish, as illustrated in figure~\ref{fig:wardidentityford}.  Moving $D$ to the boundary gives zero.\footnote{It is also interesting to consider non-scale-invariant boundary conditions. These can be interpreted as having a nontrivial operator at $\oo$.}  On the other hand, shrinking $D$ to the interior gives the sum of its actions on the individual operators.  By the Ward identity (\ref{eq:dilatationaction}), this is
\be
\label{eq:wardidentityforcorrelator}
0 &= \p{x^\mu\ptl_\mu + \Delta_1+y^\mu\ptl_\mu+\Delta_2}f(|x-y|).
\ee
We could alternatively derive (\ref{eq:wardidentityforcorrelator}) by working in some quantization, where it follows from trivial algebra and the fact that $D|0\> = 0$,
\be
0 &= \<0|[D,\cO_1(x)\cO_2(y)]|0\>\nn\\
&= \<0|[D,\cO_1(x)]\cO_2(y)+\cO_1(x)[D,\cO_2(y)]|0\>\nn\\
&= \p{x^\mu\ptl_\mu + \Delta_1+y^\mu\ptl_\mu+\Delta_2}\<0|\cO_1(x)\cO_2(y)|0\>.
\ee
Either way, the solution is
\be
f(|x-y|) &= \frac{C}{|x-y|^{\Delta_1+\Delta_2}}.
\ee


If we had an operator with negative scaling dimension, then its correlators would grow with distance, violating cluster decomposition. This is unphysical, so we expect dimensions $\De$ to be positive. Shortly, we will prove this fact for unitary conformal theories (and derive even stronger constraints on $\De$).

\begin{figure}
\begin{center}
\includegraphics[width=0.75\textwidth]{wardidentityford.jpg}
\end{center}
\caption{The Ward identity for scale invariance of a two-point function. \label{fig:wardidentityford}}
\end{figure}


\subsection{Conformal representations}

Note that $K_\mu$ is a lowering operator for dimension,
\be
D K_\mu \cO(0) &= ([D,K_\mu] + K_\mu D)\cO(0)\nn\\
&= (\De-1)K_\mu \cO(0).
\ee
(Again, we're using shorthand notation $[Q,\cO]\to Q\cO$.)  Thus, given an operator $\cO(0)$, we can repeatedly act with $K_\mu$ to obtain operators $K_{\mu_1}\dots K_{\mu_n}\cO(0)$ with arbitrarily low dimension.  Because dimensions are bounded from below in physically sensible theories, this process must eventually terminate.  That is, there must exist operators such that
\be
\label{eq:primarycondition}
[K_\mu,\cO(0)] &= 0\qquad\textrm{(primary operator)}.
\ee
Such operators are called ``primary."  Given a primary, we can construct operators of higher dimension, called ``descendants," by acting with momentum generators, which act like raising operators for dimension,
\be
\cO(0) &\to P_{\mu_1}\cdots P_{\mu_n}\cO(0)\qquad\textrm{(descendant operators)}\nn\\
\De &\to \De+n.
\ee
For example, $\cO(x)=e^{x\.P}\cO(0)$ is an (infinite) linear combination of descendant operators.
The conditions (\ref{eq:rotationatorigin},~\ref{eq:dilatationcondition},~\ref{eq:primarycondition}) are enough to determine how $K_\mu$ acts on any descendant using the conformal algebra.  For example,
\begin{exercise}
Let $\cO(0)$ be a primary operator with rotation representation matrices $\cS_{\mu\nu}$ and dimension $\Delta$.  Using the conformal algebra, show
\be
[K_\mu, \cO(x)] &= (k_\mu + 2\De x_\mu - 2x^\nu \cS_{\mu\nu})\cO(x),
\label{eq:actionofK}
\ee
where $k_\mu$ is the conformal Killing vector defined in~(\ref{eq:extraconformalgenerators}). 
\end{exercise}

To summarize, a primary operator satisfies
\be
\label{eq:isotropyaction}
\,[D,\cO(0)] &= \De\cO(0)\nn\\
\,[M_{\mu\nu},\cO(0)] &= \cS_{\mu\nu}\cO(0)\nn\\
\,[K_\mu,\cO(0)] &= 0.
\ee
From these conditions, we can construct a representation of the conformal algebra out of 
$\cO(0)$ and its descendants,
\be
\label{eq:conformalrepresentation}
\begin{array}{c|c}
\textrm{operator} & \textrm{dimension}
\\
\hline
\vdots & \\
P_{\mu_1}P_{\mu_2}\cO(0) & \De+2\\
\uparrow & \\
P_{\mu_1} \cO(0) & \De+1\\
\uparrow &\\
\cO(0) & \De.
\end{array}
\ee
The action of conformal generators on each state follows from the conformal algebra.  This should remind you of the construction of irreducible representations of $\SU(2)$ starting from a highest-weight state.  In this case, our primary is a {\it lowest-weight\/} state of $D$, but the representation is built in an analogous way.\footnote{Generically, the representation (\ref{eq:conformalrepresentation}) is an {\it induced representation} $\mathrm{Ind}^G_H(R_H)$, where $H$ is the subgroup of the conformal group generated by $D,M_{\mu\nu},K_\mu$ (called the isotropy subgroup), $R_H$ is the finite-dimensional representation of $H$ defined by (\ref{eq:isotropyaction}), and $G$ is the full conformal group. It is also called a parabolic Verma module.  Sometimes the operator $\cO$ satisfies ``shortening conditions" where a linear combination of descendants vanishes. (A conserved current is an example.)  In this case, the Verma module is reducible and the actual conformal multiplet of $\cO$ is one of the irreducible components.}  It turns out that any local operator in a unitary CFT is a linear combination of primaries and descendants. We will prove this in section~\ref{sec:onlyprimariesanddescendants}.

\begin{exercise}
Show that (\ref{commutator}), (\ref{eq:actionbyrotation}), (\ref{eq:dilatationaction}), and (\ref{eq:actionofK}) can be summarized as
\be
\label{eq:generatorsummary}
[Q_\e,\cO(x)] &= \p{\e\.\ptl + \frac{\De}{d}(\ptl\.\e) - \frac 1 2 (\ptl^\mu \e^\nu)\cS_{\mu\nu}}\cO(x).
\ee
\end{exercise}

\begin{exercise}
Deduce that $T^{\mu\nu}$ is primary by comparing (\ref{eq:generatorsummary}) with (\ref{eq:conformaltransfofT}). 
\end{exercise}

\subsection{Finite Conformal Transformations}

An exponentiated charge $U=e^{Q_\e}$ implements a finite conformal transformation.  Denote the corresponding diffeomorphism $e^{\e}$ by $x\mapsto x'(x)$. By comparing with (\ref{eq:conformalinfinitesimal}) and (\ref{eq:conformalfinite}), we find that (\ref{eq:generatorsummary}) exponentiates to
\be
\label{eq:finiteprimarytransformation}
U \cO^a(x) U^{-1} &= \Omega(x')^\De D(R(x'))_b{}^a\cO^b(x'),
\ee
where as before
\be
\pdr{x'^\mu}{x^\nu} &= \Omega(x')R^\mu{}_\nu(x'),\qquad R^\mu{}_\nu(x')\in\SO(d).
\ee
Here, $D(R)_b{}^a$ is a matrix implementing the action of $R$ in the $\SO(d)$ representation of $\cO$, for example
\begin{align}
D(R) &= 1 && \textrm{(scalar representation)},\nn\\
D(R)_\mu{}^\nu &= R_\mu{}^\nu && \textrm{(vector representation)},\label{eq:vectorrep}\nn\\
 &\cdots&& \cdots
\end{align}
and so on.

We could have started the whole course by taking (\ref{eq:finiteprimarytransformation}) as the definition of a primary operator. But the connection to the underlying conformal algebra will be crucial in what follows, so we have chosen to derive it.

\begin{exercise}
Show that the transformation (\ref{eq:finiteprimarytransformation}) composes correctly to give a representation of the conformal group.  That is, show
\be
U_{g_1}U_{g_2}\cO^a(x) U_{g_2}^{-1} U_{g_1}^{-1} &= U_{g_1g_2}\cO^a(x)U_{g_1g_2}^{-1}
\ee
where $x\mapsto g_{i}(x)$ are conformal transformations, $g_1g_2$ denotes composition $x\mapsto g_1(g_2(x))$, and $U_{g}$ is the unitary operator associated to $g$.
\end{exercise}

\section{Conformal correlators}

\subsection{Scalar operators}
\label{sec:conformalcorrelatorsscalars}

We have already seen that scale invariance fixes two-point functions of scalars up to a constant
\be
\label{eq:scaletwoptfunction}
\<\cO_1(x_1)\cO_2(x_2)\> &= \frac{C}{|x_1-x_2|^{\De_1+\De_2}} \qquad\textrm{(SFT)}.
\ee

For primary scalars in a CFT, the correlators must satisfy a stronger Ward identity,
\be
\label{eq:scalarconformalcorrelator}
\<\cO_1(x_1)\dots\cO_n(x_n)\> &= \<(U\cO_1(x_1)U^{-1}) \cdots (U\cO_n(x_n)U^{-1})\>\nn\\
 &= \Omega(x_1')^{\De_1}\cdots\Omega(x_n')^{\De_n}\<\cO_1(x_n')\cdots \cO_n(x_n')\>.
\ee
Let us check whether this holds for (\ref{eq:scaletwoptfunction}).

\begin{exercise}
Show that for a conformal transformation,
\be
\label{eq:conformaltransformationofdistance}
(x-y)^2 &= \frac{(x'-y')^2}{\Omega(x')\Omega(y')}.
\ee
Hint: This is obviously true for translations, rotations, and scale transformations. It suffices to check it for inversions $I:x\to\frac{x}{x^2}$ (why?).
\end{exercise}
Using (\ref{eq:conformaltransformationofdistance}), we find
\be
\frac{C}{|x_1-x_2|^{\De_1+\De_2}} &= \Omega(x_1')^{\frac{\De_1+\De_2}{2}}\Omega(x_2')^{\frac{\De_1+\De_2}{2}}\frac{C}{|x_1'-x_2'|^{\De_1+\De_2}}.
\ee
Consistency with (\ref{eq:scalarconformalcorrelator}) then requires $\De_1=\De_2$ or $C=0$.  In other words,
\be
\<\cO_1(x_1)\cO_2(x_2)\> &= \frac{C\de_{\De_1\De_2}}{x_{12}^{2\De_1}}\qquad\textrm{(CFT, primary operators)},
\ee
where $x_{12}\equiv x_1-x_2$.
\begin{exercise}
Recover the same result using the Ward identity for $K_\mu$
\be
\<[K_\mu,\cO_1(x_1)]\cO_2(x_2)\>+\<\cO_1(x_1)[K_\mu,\cO_2(x_2)]\> &= 0.
\ee
\end{exercise}

Conformal invariance is also powerful enough to fix a three-point function of primary scalars, up to an overall coefficient.  Using (\ref{eq:conformaltransformationofdistance}), it's easy to check that the famous formula \cite{Polyakov:1970xd}
\be
\label{eq:conformalthreeptfunction}
\<\cO_1(x_1)\cO_2(x_2)\cO_3(x_3)\> &= \frac{f_{123}}{x_{12}^{\De_1+\De_2-\De_3}x_{23}^{\De_2+\De_3-\De_1}x_{31}^{\De_3+\De_1-\De_2}},
\ee
with $f_{123}$ constant, satisfies the Ward identity (\ref{eq:scalarconformalcorrelator}).

With four points, there are nontrivial conformally invariant combinations of the points called ``conformal cross-ratios,"
\be
\label{eq:definitionofcrossratios}
u = \frac{x_{12}^2 x_{34}^2}{x_{13}^2 x_{24}^2},\qquad
v = \frac{x_{23}^2 x_{14}^2}{x_{13}^2 x_{24}^2}.
\ee
The reason that there are exactly two independent cross-ratios can be understood as follows.
\begin{itemize}
\item Using special conformal transformations, we can move $x_4$ to infinity.
\item Using translations, we can move $x_1$ to zero.
\item Using rotations and dilatations, we can move $x_3$ to $(1,0,\dots,0)$.
\item Using rotations that fix $x_3$, we can move $x_2$ to $(x,y,0,\dots,0)$.
\end{itemize}

\begin{figure}
\begin{center}
\includegraphics[width=0.4\textwidth]{fig-z}
\end{center}
\caption{\label{fig:zplane} Using conformal transformations, we can place four points on a plane in the configuration shown above (figure from \cite{Hogervorst:2013sma}).}
\end{figure}

This procedure leaves exactly two undetermined quantities $x,y$, giving two independent conformal invariants. Evaluating $u$ and $v$ for this special configuration of points (figure~\ref{fig:zplane}) gives
\be
u=z\bar z,\qquad v=(1-z)(1-\bar z),
\ee
where $z\equiv x+iy$.

Four-point functions can depend nontrivially on the cross-ratios.  For a scalar $\f$ with dimension $\De_\phi$, the formula
\be
\label{eq:fourptfunctionofprimaries}
\<\f(x_1)\f(x_2)\f(x_3)\f(x_4)\> &= \frac{g(u,v)}{x_{12}^{2\De_\f}x_{34}^{2\De_\f}}
\ee
satisfies the Ward identity (\ref{eq:scalarconformalcorrelator}) for any function $g(u,v)$. 
\begin{exercise}
Generalize (\ref{eq:fourptfunctionofprimaries}) to the case of non-identical scalars $\f_i(x)$ with dimensions $\De_i$.
\end{exercise}

The left-hand side of (\ref{eq:fourptfunctionofprimaries}) is manifestly invariant under permutations of the points $x_i$.  This leads to consistency conditions on $g(u,v)$,
\begin{align}
\label{eq:trivialcrossing}
g(u,v) &= g(u/v,1/v) && \textrm{(from swapping $1\leftrightarrow 2$ or $3\leftrightarrow 4$)},\\
\label{eq:crossingsymmetry}
g(u,v) &= \p{\frac{u}{v}}^{\De_\f} g(v,u) && \textrm{(from swapping $1\leftrightarrow 3$ or $2\leftrightarrow 4$)}.
\end{align}
All other permutations can be generated from the ones above.  We will see shortly that $g(u,v)$ is  actually determined in terms of the dimensions $\De_i$ and three-point coefficients $f_{ijk}$ of the theory.  Equation~(\ref{eq:trivialcrossing}) will be satisfied for trivial reasons.  However (\ref{eq:crossingsymmetry}) will lead to powerful constraints on the $\De_i, f_{ijk}$.



\begin{appendix}
\end{appendix}

\bibliographystyle{ws-rv-van}
\bibliography{ph229-notes}
\ifarxivsubmission
\else
  \blankpage
\fi
\printindex
\end{document}
